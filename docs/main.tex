\documentclass[usenatbib]{mnras}
\usepackage{bm}
\usepackage{amsmath}
\usepackage{booktabs}
\usepackage{graphicx}
\usepackage{comment}


\newcommand{\setof}[1]{\left\{{#1}\right\}}
\newcommand{\given}{\,|\,}
\newcommand{\dd}{\mathrm{d}}
\newcommand{\catalog}{\bm{Q}}
\newcommand{\pars}{\bm{\theta}}
\newcommand{\amin}{\ifmmode {^{\prime}\ }\else$^{\prime}$\fi}
\newcommand{\asec}{\ifmmode {^{\prime\prime}}\else$^{\prime\prime}$\fi}
\newcommand{\bs}[1]{\boldsymbol{#1}}

\newcommand{\Msun}{\ifmmode {M_{\odot}}\else${M_{\odot}}$\fi}
\newcommand{\Porb}{\ifmmode {P_{\rm orb}}\else${P_{\rm orb}}$\fi}
\newcommand{\sse}{{\tt SSE}}
\newcommand{\bse}{{\tt BSE}}
\newcommand{\dart}{{\tt dart\_board}}



\title[Pop. synth. with MCMC]{Binary population synthesis with MCMC}
\author[J. J. Andrews et al.]{Jeff J. Andrews$^{1,2}$\thanks{Contact e-mail: \href{mailto:andrews@physics.uoc.gr}{andrews@physics.uoc.gr}}, Andreas Zezas$^{1,2,3}$, Tassos Fragos$^{4}$ \\
$^1$ Foundation for Research and Technology-Hellas, 71110 Heraklion, Crete, Greece \\
$^2$ Physics Department \& Institute of Theoretical \& Computational Physics, University of Crete, 71003 Heraklion, Crete, Greece \\
$^3$ Harvard-Smithsonian Center for Astrophysics, 60 Garden Street, Cambridge, MA 02138, USA \\
$^4$ Geneva Observatory, University of Geneva, Chemin des Maillettes 51, 1290 Sauverny, Switzerland }

\begin{document}
\label{firstpage}
\pagerange{\pageref{firstpage}--\pageref{lastpage}}
\maketitle

\begin{abstract}
By employing Monte Carlo random sampling, traditional binary population synthesis (BPS) offers a substantial improvement over brute force, grid-based studies. Even so, BPS models typically require a large number of simulation realizations, a computationally expensive endeavor, to generate statistically robust results. For certain problems, even this type of sampling is limited by its computational cost. In this work we describe a novel approach which treats the initial binary parameters as model parameters and uses a Markov Chain Monte Carlo technique to explore the valid region of parameter space, given the properties of an observed population. As a key aspect of this approach, we include the supernova kick magnitude and direction as model parameters. In addition to being substantially more efficient than traditional BPS methods for certain types of binaries, this approach builds in the fit to observations. We describe our publicly available code \dart, which is a statistical wrapper for rapid binary population synthesis codes. This method can be flexibly adapted to a variety of stellar binary populations, and we apply our code to four test cases: (i) the population of high mass X-ray binaries (HMXBs), (ii) the population of HMXBs in the Large Magellanic Cloud (LMC) in which the spatially resolved star-formation history is used as a prior, (iii) one particular HMXB in the LMC, Swift J0513.4$-$6547, in which we include observations of the system's component masses and orbital period, and (iv) the population of stellar-mass binary black holes detected by gravitational wave observatories. 
\end{abstract}

\begin{keywords}
binaries: close, X-rays: binaries, X-rays: galaxies, stars: statistics, Magellanic Clouds
\end{keywords}

\section{Introduction}
With its unprecedented angular resolution, the space-based X-ray observatory {\it Chandra} has identified hundreds of X-ray point sources in nearby galaxies \citep{sarazin01,ho01,fabbiano01,fabbiano06}. Studies of these objects have yielded a deeper insight into the physical processes forming individually identified stellar sources, including low and high mass X-ray binaries \citep[LMXBs and HMXBs, respectively;][]{lehmer10}, as well as ultra-luminous X-ray sources \citep[ULX;][]{swartz04,feng11}. HMXBs in particular are formed from the accretion of material from an early-type star onto either a neutron star or black hole \citep[for a review, see][]{bhattacharya91}. Their relatively short lifetimes imply that HMXBs are indicators of recent star-formation, and indeed extragalactic observations find that the contribution from HMXBs to the collective X-ray luminosity of a galaxy increases with increasing star-formation rate \citep{grimm03,lehmer10,mineo12}. Furthermore, observations have shown that within nearby, resolved galaxies, HMXBs and ULXs are typically found near regions with recent star-formation \citep{zezas02a, kaaret04, antoniou10}.


The nearby Small Magellanic Cloud (SMC) and Large Magellanic Cloud (LMC) have the best studied extragalactic X-ray populations. X-ray campaigns with {\it Chandra} \citep[e.g.,][]{laycock10} and {\it XMM-Newton} \citep[e.g.,][]{sturm13} have brought the number of candidate and confirmed HMXBs in the SMC to 148 \citep[for the most recent catalog, see][]{haberl16}, and ongoing observations are identifying and characterizing new X-ray objects in the somewhat larger LMC \citep{antoniou16}. At the same time, infrared, optical, and ultraviolet imaging provide detailed spatially resolved star-formation histories of regions within the SMC and LMC, with angular resolutions as small as 12\amin\ by 12\amin\ \citep{harris04,harris09}. These star-formation histories are precise, particularly so in the past 10$^8$ yr when HMXBs were formed.


Models describing the HMXB population in the SMC have primarily focused on the effect from the SMC's sub-solar metallicity on HMXB orbital parameters \citep{dray06} and X-ray luminosity function \citep{linden10}. Separate models comparing regions of past star-formation to the observed HMXBs have focused on the distance a population of HMXBs is expected to travel as a function of time \citep{sepinsky05} and how that distance can correlate with binary characteristics such as orbital period and X-ray luminosity \citep{zuo10, zuo15}. 


These studies all rely on traditional binary population synthesis (BPS) in which one randomly generates, according to some predetermined initial probability distributions, a large number of stellar binaries. These binaries are then evolved using a rapid binary evolution code until their present state, when one then takes a `snapshot' of those binaries that currently exist as HMXBs; evolutionary processes such as supernova (SN) kicks and common envelope evolution disrupt or merge most of the binaries initially generated \citep[For a recent discussion of state-of-the-art BPS codes and their differences, see;][]{toonen14}. For rare or short-lived evolutionary states, BPS is an inefficient tool since significant computational time is spent on regions of parameter space of no interest to the observed systems. Therefore, in many BPS studies, more than $10^6$ binaries are often required to make statistically robust comparisons with observed populations. For instance, in a recent study of the ULX M82 X-2 \citet{fragos15} generated $10^7$ initial binaries, finding only $10^2-10^3$ evolved into systems matching the observations of M82 X-2. An alternative is desired.


One approach uses a Jacobian formalism to transform initial binary probability distributions to distributions of observed parameters. This method has been developed by \citet{kolb93} and \citet{politano96} for cataclysmic variables and extended by \citet{kalogera96} for high mass binaries including SN kicks \citep[see also][]{kalogera98, kalogera00}. Jacobian transformations have most recently been employed by \citet{bhadkamkar12,bhadkamkar14} to describe populations of HMXBs and LMXBs, respectively. Although analytic and efficient, these most recent works lack the flexibility to account for spatially resolved star-formation histories, and it remains to be seen whether analytic methods can incorporate the level of detail required to provide more than qualitative comparisons with observations.


\newpage

Many groups have recently developed methods to combine population synthesis with Bayesian inference, including matching stellar populations with isochrones \citep{stenning16}, deriving galactic properties from photometry and spectra \citep{krumholz15}, and obtaining photometric redshifts \citep{tanaka15}, among others. Specific to HMXBs, \citet{douna15} developed a Markov Chain Monte Carlo (MCMC) approach to correlate the X-ray luminosity function from a population of HMXBs with a galaxy's star-formation rate and metallicity. Previous works by \citet{ihm06} and \citet{andrews15} each used Bayesian statistical techniques in post-processing to compare traditional BPS results with observations of double neutron stars.


In this work, we describe \dart, an open-source code that provides a statistical wrapper to rapid binary population synthesis codes. 
%In this work, we develop a fully Bayesian approach that models the formation of binary stars HMXBs in the SMC using an MCMC technique. 
We consider the initial binary parameters (including the SN kick magnitude and direction) as model parameters with prior probabilities based on the same initial distributions used by traditional BPS. We demonstrate how the star-formation history (either spatially resolved or not) can be used as a prior on the birth position and time. This increases the constraining power of our method since it incorporates the current positions of binaries in the fit. Our likelihood function flexibly combines available binary observables, which allows our method to be adaptable to model either individual systems or a population. Crucially, because MCMC focuses computation power based on the posterior probability (rather than the prior distributions as in traditional population synthesis), little computation time is wasted on evolving binaries that disrupt or merge. The gain in efficiency can be substantial. Using a modified version of the widely used rapid binary evolution code {\tt BSE} \citep{hurley00,hurley02} with {\tt python} bindings, we demonstrate the viability of this method. 


In Section \ref{sec:stats} we describe our statistical method for and provide the relevant prior and posterior distributions. We test our model on four separate samples in Section \ref{sec:results}. Finally, we place our method in the broader context of binary population studies, providing some limitations and future directions as well as our conclusions in Section \ref{sec:discussion}.








\section{Statistical Method}
\label{sec:stats}

In this section we define our statistical method, with specific attention to the differences between our method and traditional population synthesis. In Section \ref{sec:stats_population} and \ref{sec:stats_individual} we define how our method can be used to model populations of binaries and individual binaries, respectively. Following that, we describe our prior probabilities in Section \ref{sec:priors} and our likelihood functions in Section \ref{sec:likelihoods}.


\subsection{Modeling Populations}
\label{sec:stats_population}


BPS randomly produces a set of binaries by evolving the distribution of initial binary parameters through binary evolution prescriptions. Since in general we do not {\it a priori} know which initial conditions will produce systems of a certain population (for intance, many systems are disrupted or merge during their evolution), we must test the entire region of initial binary parameter space that could plausibly produce that population. Traditional binary BPS solves this problem by making random draws of $\vec{x}_{\rm i}$, the initial binary parameters, from observationally-derived distribution functions, $P(\vec{x}_{\rm i})$:
\begin{equation}
\vec{x}_{\rm i} \sim P(\vec{x}_{\rm i}).
\end{equation}
To first order, high mass binaries can be determined uniquely by only a few parameters: the binary's initial masses, $M_{\rm 1,i}$ and $M_{\rm 2,i}$, its separation, $a_{\rm i}$, and eccentricity, $e_{\rm i}$, the kick velocity it received after the primary collapsed to form a NS or BH, $\vec{v}_{\rm k}$, and its birth time $t_{\rm i}$. Depending on the population being modeled, we may optionally include the coordinates for the binary's birth position, $\alpha_{\rm i}$ and $\delta_{\rm i}$, and the kick velocity received when the secondary collapsed to form a compact object:
\begin{equation}
\vec{x}_{\rm i} \equiv ( M_{\rm 1,i}, M_{\rm 2,i}, a_{\rm i}, e_{\rm i}, \vec{v}_{\rm k, SN1}, \vec{v}_{\rm k, SN2}, \alpha_{\rm i}, \delta_{\rm i}, t_{\rm i} ). \label{eq:x_i}
\end{equation}
One key aspect of this model is that the SN kick magnitude and direction is included as model parameters rather than determined from random draws on-the-fly during binary evolution.



Using a binary evolution code, these initial binaries are then evolved from $\vec{x}_{\rm i}$ into its current state, represented by $\vec{x}_{\rm f}$:
\begin{equation}
\vec{x}_{\rm f} = f(\vec{x}_{\rm i}). \label{eq:xf_xi}
\end{equation}
We can now define a function $P(x_{\rm type} \given \vec{x}_{\rm i})$ which is either unity or zero depending on whether $\vec{x}_{\rm f}$ represents a system of a specific type:
\begin{equation}
P(x_{\rm type} | \vec{x}_{\rm i}, M) = 
\begin{cases}
1, & \vec{x}_{\rm f} \in x_{\rm type} \\
0, & \vec{x}_{\rm f} \notin x_{\rm type},
\end{cases}
\end{equation}
where $M$ is our binary evolution model. Distributions of the components of $\vec{x}_{\rm f}$ (such as the spatial distribution of systems, X-ray luminosity function of HMXBs, orbital period distribution, etc.) can then provide model predictions for populations or comparisons to observational samples.




For binary populations involving NSs and black holes (BH), traditional BPS may be an inefficient tool; mass transfer and SN kicks may merge or disrupt the majority of systems before they evolve into objects of interest (i.e., $P(x_{\rm type} \given \vec{x}_{\rm i}) = 0$ for many or even most of the randomly drawn $\vec{x}_{\rm i}$). The fact that these binaries are discarded is a major source of computational expense in traditional BPS. 







% MCMC population synthesis
Instead of random draws of $\vec{x}_{\rm i}$, we consider the components of $\vec{x}_{\rm i}$ model parameters. In this formulation $P(\vec{x}_{\rm i})$ is the prior probability on the model parameters, and $P(x_{\rm type} \given \vec{x}_{\rm i})$ is the likelihood of producing a binary of a particular type from a given $\vec{x}_{\rm i}$. Using Bayes' Theorem, we can then identify the set of $\vec{x}_{\rm i}$ most likely to produce these binaries:
\begin{equation}
P(\vec{x}_{\rm i} \given x_{\rm type}) = \frac{P(x_{\rm type} \given \vec{x}_{\rm i}) P(\vec{x}_{\rm i})} {P(x_{\rm type})}. \label{eq:bayes_pop}
\end{equation}
We ignore $P(x_{\rm type})$, which serves as a normalization constant, and define the posterior probability as the numerator on the right hand side of Equation \ref{eq:bayes_pop}. Although in this work we ignore the denominator, it could be used to account for observational biases in future studies. The large dimensionality of $\vec{x}_{\rm i}$ argues for an efficient numerical method to probe the region of viable parameter space.


In an MCMC code, a `walker' moves around the $\vec{x}_{\rm i}$ parameter space: the posterior probability of the current $\vec{x}_{\rm i}$ is calculated, a new trial $\vec{x}_{\rm i}$ is randomly selected, the posterior probability of the new position is compared to that of the current position, and depending on the ratio of the two posterior probabilities, the new $\vec{x}_{\rm i}$ is either selected and added to the chain or rejected and the current position is kept for another step. The chain stores a record of all the walker's past positions. Samples from this chain comprise the synthetic population analogous to the population generated by traditional BPS.\footnote{Since the current walker position will necessarily be closely related to the previous step, the posterior samples will be correlated with some characteristic length. The autocorrelation length needs to be calculated {\it a posteriori}, and only one sample per autocorrelation length can be considered independent. We return to this in Section \ref{sec:limitations}.}


In principle, for an infinite number of iterations, the distribution of posterior samples of $\vec{x}_{\rm i}$ produced by this method will identically mimic the distribution generated by traditional BPS. Since we are limited to a finite sample, the computation time to produce a statistically robust sample using each method depends on the relative formation efficiency of systems and the autocorrelation length of the MCMC posterior distribution. For high formation efficiencies, traditional methods are preferred, since every random draw from BPS is independent. However for systems with low formation efficiencies or short lifetimes, MCMC is the preferred choice.


Correct implementation requires careful attention to the prior distributions, $P(\vec{x}_{\rm i})$, and an efficient method to calculate $P(x_{\rm type} \given \vec{x}_{\rm i})$. We describe how we calculate the prior probabilities in Section \ref{sec:priors} and our single and binary evolution prescriptions in Sections \ref{sec:single_star} and \ref{sec:binary_evolve}, respectively.






% % SSE
% \subsection{Single Star Evolution} \label{sec:single_star}

% Our single stellar evolutionary model is based on the stellar evolution fitting formulae, \sse\ \citep{hurley00}. Rather than using function calls within our code to \sse, we pre-compute a series of \sse\ models ranging in initial mass from 1.0 to 40.0 \Msun, with a spacing of 0.01 \Msun. These models use default parameters and a metallicity of 0.008 to match that of the SMC star-formation maps. 

% From the evolutionary history computed by \sse\ for each star (typically 200-500 unequally-spaced time steps each), we use {\tt scipy}'s {\tt interp1d} routine to generate linear interpolations across the lifetimes of these stars. We separately interpolate as a function of time the key quantities: radius, mass, and mass loss rate. We then combine each of these interpolations into an array. For any arbitrary mass star, to obtain quantities of interest, we round down to the next lowest mass interpolation and access the corresponding interpolated curve. This method allows for fast calculation of fundamental properties of a star at an input age, while providing a sufficiently smooth transition over both mass and time.

% For each star these models additionally provide the He core mass, maximum radius, stellar lifetime and H burning lifetime. We create additional interpolations as a function of initial mass for these quantities.

% In addition to hydrogen star models, we pre-compute a separate series of {\tt sse} runs to model so-called naked helium star evolution, again for a metallicity of $Z=0.008$. We apply the updated He-star wind prescription found in \citep{belczynski10}. These models indicate that the mass lost in stellar winds prior to the star's core collapse can be substantial. We generate an interpolation curve of the final He-star mass prior to SN as a function of the initial He-star mass.


% Binary evolution
% \subsection{Binary Evolution} \label{sec:binary_evolve}

% In this study, we only consider one evolutionary channel forming wind-fed NS-HMXBs. We further ignore many binary dynamical effects such as tidal evolution prior to the initial phase of mass transfer and during the X-ray luminous phase, which we only expect to be significant for the minority of HMXBs in which the donor is a supergiant. We use a simplified binary prescription scheme here to showcase how our MCMC method works.

% After its initial state, there are three major stages that we include in our binary evolution prescriptions. First, the more massive primary evolves first off the main sequence, leading to stable, thermal-time-scale mass transfer onto the secondary. As a helium star, the primary can lose a substantial fraction of its mass to the interstellar medium in a strong stellar wind. After the primary has completed its evolution, it undergoes core-collapse forming a NS. The system suffers effectively instantaneous mass loss and a natal kick, both of which act to profoundly affect the binary's orbit. Finally, the system begins to emit in X-rays once the NS accretes from the secondary's stellar wind. The companion's mass and mass loss rate are both a function of time since they change as the system evolves. We discuss each of these three evolutionary transitions below.


% % Thermal timescale mass transfer
% \subsubsection{The First Mass Transfer Phase} \label{sec:trans_MT}

% Once the first star evolves past its main sequence onto the giant branch, it begins overfilling its Roche lobe. We ignore binaries with large enough initial separations that avoid this phase of mass transfer. Our prior probability on $M_{\rm 2,i}$ selects for only those systems that will undergo stable mass accretion. As discussed in Section \ref{sec:priors} systems in our model with mass-ratios below 0.3 will enter unstable mass transfer. The HMXB formation channel where the binary goes through a common envelope, prior to the compact object formation, is not considered in this study. Since we are considering only wind-fed HMXBs, which tend to have longer orbital periods compared to their Roche-lobe overfilling counterparts, the contribution of the aforementioned channel is expected to be negligible. 

% We assume that this initial phase of stable mass transfer instantaneously circularizes the orbit at the pericenter separation: $a_{\rm i} (1-e_{\rm i})$. The post mass transfer primary becomes the primary core mass ($M_{\rm 1,c}$), while the secondary incorporates the primary's lost envelope. We assume conservative mass transfer:
% \begin{eqnarray} 
% M_{\rm 1, MT} &=& M_{\rm 1,c} \\
% M_{\rm 2, MT} &=& M_2 + M_1 - M_{\rm 1,c},
% \end{eqnarray}
% $M_{\rm 1,c}$ is determined by the interpolations between the \sse\ models described in Section \ref{sec:single_star}. The post-mass transfer orbital separation is therefore:
% \begin{equation}
% a_{\rm MT} = a_{\rm i} (1-e_{\rm i}) \left[ \frac{M_{\rm 1,i} M_{\rm 2,i}}{M_{\rm 1, MT} M_{\rm 2, MT}} \right]^2.
% \end{equation}



% From the first mass transfer phase, the secondary gained a significant amount of hydrogen, which, depending on the initial stellar parameters, could more than double its mass. Clearly such mass gain will affect the stellar lifetime of the secondary, and for single stellar evolution models to provide correct parameters for the secondary, we need to use an adjusted age. 

% The first mass transfer phase will occur at roughly the He ignition time, $t_{\rm He} (M_{\rm 1,i})$, for a star of mass $M_{\rm 1,i}$. If we assume that stars burn H at a constant rate throughout their MS, the mass of H converted into He in the secondary at this point will be:
% \begin{equation}
% M_{\rm He,2} = \frac{t_{\rm He}(M_{\rm 1,i})}{t_{\rm He}(M_{\rm 2,i})} X M_{\rm He,core}(M_{\rm 2,i}), \label{eq:M_He}
% \end{equation}
% where $X$ is the initial H fraction and $M_{\rm He,core}(M_{\rm 2,i})$ is the He core mass (at He ignition) of a star with an initial mass of $M_{\rm 2,i}$. At this point, the secondary will accrete mass from the primary with an identical H fraction $X$. This mass transfer is assumed to be instantaneous since it occurs on a thermal time-scale, significantly shorter than the stellar lifetime. The rejuvenated secondary star now has a substantially different mass ($M_{\rm 2, MT}$), but it is not beginning at zero age. We can estimate its effective age immediately after mass transfer has completed ($t_{\rm eff}$) by comparing the mass already converted into He from Equation \ref{eq:M_He} with the total mass of H a star of mass $M_{\rm 2, MT}$ will burn on the MS, $X M_{\rm He,core}(M_{\rm 2, MT})$, multiplied by the He ignition time:
% \begin{equation}
% t_{\rm eff} = \frac{M_{\rm He, 2}}{X M_{\rm He, core} (M_{\rm 2, MT}) } t_{\rm He} (M_{\rm 2, MT}). \label{eq:t_eff_1}
% \end{equation}
% The secondary's effective age at the time of observation is $t_{\rm eff,obs} = t_{\rm eff} + t_i - t_{\rm He}(M_{\rm 1,i})$, which can be expressed using Equations \ref{eq:M_He} and \ref{eq:t_eff_1}:
% \begin{eqnarray}
% t_{\rm eff,obs} &=& \frac{M_{\rm He,core}(M_{\rm 2,i})}{M_{\rm He,core}(M_{\rm 2, MT})} \frac{t_{\rm He}(M_{\rm 1,i})}{t_{\rm He}(M_{\rm 2,i})} t_{\rm He}(M_{\rm 2, MT}) \nonumber \\
%  & & \qquad + t_{\rm i} - t_{\rm He}(M_{\rm 1,i}). \label{eq:t_eff_obs_full}
% \end{eqnarray}
% \citet{tout97} provide a similar equation for the rejuvenation of a star due to mass accretion. Using $t_{\rm eff,obs}$ in the interpolations, we can determine the adjusted mass, radius, and wind mass loss rate of the secondary using single stellar evolution models. 



% \subsubsection{The Primary's Core Collapse} \label{sec:trans_SN}

% Prior to its core collapse, the primary star has survived for some time as a naked He-star with substantially enhanced winds \cite{belczynski10}. We ignore any accretion of these winds by the secondary, so mass loss tends to expand the binary's orbit. Our {\tt sse} interpolation provides the pre-SN He-star mass, $M_{\rm pre-SN}$ as a function of $M_{\rm MT}$. Using Jeans mode mass loss we can determine the pre-SN orbital separation, $a_{\rm pre-SN}$:
% \begin{equation}
% a_{\rm pre-SN} = \frac{M_{\rm 1, MT} + M_{\rm 2, MT}}{M_{\rm 1, pre-SN} + M_{\rm 2, MT}} a_{\rm MT}
% \end{equation}


% We next calculate the post-SN orbital separation, $a_{\rm SN}$, systemic velocity, $v_{\rm sys}$, and eccentricity, $e_{\rm SN}$, based on the equations in \citet{hills83} and \citet{kalogera96}. In our model, all core collapsing stars form 1.35 \Msun\ NSs. \citet{timmes96} suggest the maximum initial mass forming a NS is $\sim$20 \Msun, instead of the maximum mass of 30 \Msun\ we choose in this work; however as we will see below, our results are largely independent of this choice. Energy conservation provides $a_{\rm SN}$:
% \begin{equation}
% a_{\rm SN} = \left[ \frac{2 }{a_{\rm pre-SN}}  - \frac{v_1^2}{G(M_{\rm NS} + M_{\rm 2, MT})} \right]^{-1}. \label{eq:SN_A} \\
% \end{equation}
% where $v_1$ is the post-kick velocity of the primary (in the reference frame of an initially stationary secondary):
% \begin{equation}
% v_1^2 = 2v_{\rm k} v_{\rm r} \cos \theta_{\rm k} + v_{\rm k}^2 + v_{\rm r}^2, \label{eq:v_1}
% \end{equation}
% where $\theta_{\rm k}$ defines the angle between the kick velocity and the direction of orbital motion. Here, we have made the additional substitution:
% \begin{equation}
% v_{\rm r} = \sqrt{\frac{G (M_{\rm 1, pre-SN} + M_{\rm 2, MT})}{a_{\rm pre-SN}}}. \label{eq:v_r}
% \end{equation}
% The systemic velocity is:
% \begin{equation}
% v_{\rm sys}^2 = \beta^2 v_{\rm k}^2
%   + v_{\rm r}^2 \left( \alpha - \beta \right)^2
%   + 2 \beta v_{\rm r} v_{\rm k} \cos \theta_{\rm k} \left( \alpha - \beta \right)
%     \label{eq:SN_v_sys}
% \end{equation}
% where we have included two substitutions:
% \begin{eqnarray}
% \alpha &=& \frac{M_{\rm 1, pre-SN}}{M_{\rm 1, pre-SN} + M_{\rm 2, MT}} \\
% \beta &=& \frac{M_{\rm NS}}{M_{\rm NS} + M_{\rm 2, MT}}
% \end{eqnarray}


% The post-SN eccentricity is determined by angular momentum conservation:
% \begin{eqnarray}
% 1-e_{\rm SN}^2 &=& \frac{a_{\rm pre-SN}^2}{a_{\rm SN}\ G (M_{\rm NS} + M_{2, MT})} \nonumber \\
%  & & \times \left( v_{\rm k}^2 \cos^2\theta_{\rm k} + v_{\rm k}^2 \sin^2 \theta_{\rm k} \sin^2 \phi_{\rm k} \right. \nonumber \\
%  & & \qquad + \left. 2 v_{\rm k} v_{\rm r} \cos \theta_{\rm k} + v_{\rm r}^2  \right). \label{eq:SN_e}
% \end{eqnarray}
% Systems with $0 \leq e < 1$ remain bound. It should be noted that $\phi_{\rm k}$ only affects the post-SN orbit through the $\sin^2 \phi_{\rm k}$ term when solving for the post-SN orbital eccentricity in Equation \ref{eq:SN_e}. 



% \subsubsection{The X-ray Luminous Phase} \label{sec:trans_XRB}

% In the final stage of evolution, the NS accretes mass from the stellar wind of the massive secondary. Although we do not calculate the X-ray luminosity of the system, for it to be observed as an HMXB, accretion must be non-zero; if the secondary has completely evolved, the system is not observed as an HMXB. We ignore the effect of tides during this phase of evolution, but we do include the expansion of the orbit based on mass loss from the system:
% \begin{eqnarray}
% a_{\rm f} &=& \frac{M_{\rm NS} + M_{\rm 2, SN}}{M_{\rm NS} + M_{\rm 2,f}} a_{\rm SN} \\
% e_{\rm f} &=& e_{\rm SN}.
% \end{eqnarray}
% Here, we have assumed that the NS accretes a negligible fraction of the secondary's wind. Therefore, the secular evolution of the orbit is very close to the fully non-conservative limit.





% Individual HMXBs
\subsection{Modeling Individual HMXBs} \label{sec:stats_individual}

If we would like to quantitatively compare a model to a set of observationally derived properties, $D$, of a particular system rather than a population, we need to adapt our method. We may be interested in either deriving the initial binary conditions that could have produced the observed systems, $P(\vec{x}_{\rm i} \given D)$, or determining the current binary parameters, $P(\vec{x}_{\rm f} \given D)$. These two quantities are closely related since binary evolution directly relates $\vec{x}_{\rm i}$ to $\vec{x}_{\rm f}$. 



Instead of calculating $P(\vec{x}_{\rm i} \given x_{\rm HMXB})$ as in Equation \ref{eq:bayes_pop}, we can again use Bayes' Theorem to calculate $P(\vec{x}_{\rm i} \given D)$:
\begin{equation}
P(\vec{x}_{\rm i} \given D) = \frac{P(D \given \vec{x}_{\rm i})\ P(\vec{x}_{\rm i})}{P(D)}, \label{eq:bayes_ind}
\end{equation}
where $P(D \given \vec{x}_{\rm i})$ is the likelihood function, $P(\vec{x}_{\rm i})$ is the prior probabilities on the model parameters, and $P(D)$ is again a normalization constant that for our purposes can be ignored. The posterior probability is the numerator in the right hand side of Equation \ref{eq:bayes_ind}.



Traditional BPS takes a shotgun approach, making many random draws of $\vec{x}_{\rm i}$ from $P(\vec{x}_{\rm i})$. If one wants to then calculate a Bayesian likelihood, $P(\vec{x}_{\rm i} \given D)$ can be determined from those systems that are consistent with $D$ \citep{andrews15}. Results are then derived from the selected subset of systems. Since only a small subset of the binaries that form HMXBs will be consistent with any particular observed system, the likelihood function may be non-zero for only a small region of parameter space. The more precisely a system is measured, the smaller the phase space volume of interest. 






We simulate individual HMXBs using the same model parameters that we use for a population of HMXBs, so $\vec{x}_{\rm i}$ is determined by Equation \ref{eq:x_i}. The priors on these parameters, which we discuss in Section \ref{sec:priors_indiv}, are somewhat different for individual systems, compared with those derived for populations of HMXBs.

Individual HMXBs in the SMC may have well measured $P'_{\rm orb}$, $e'$, and $M'_2$ (primed quantities indicate observed rather than true, underlying quantities) determined from X-ray timing and observations of the optical counterpart to the X-ray source. Each of these measured quantities has some uncertainty associated with it which should be taken into account. Furthermore, individual HMXBs have a specific location that we are trying to associate with nearby star forming regions. We start by defining $D$ as:
\begin{equation}
D \equiv (\alpha, \delta, P'_{\rm orb}, e', M'_2). \label{eq:D}
\end{equation}
Uncertainties on the primed quantities are not explicitly included in $D$, and we ignore uncertainties on the position.


To generate our likelihood function, we now marginalize over four parameters, the true orbital parameters, $P_{\rm orb}$ and $e$, the true (current) companion mass $M_2$, and the systemic velocity ($v_{\rm sys}$). Our model likelihood then becomes:
\begin{eqnarray}
P(D \given \vec{x}_{\rm i}, M) &=& \int \dd P_{\rm orb}\ \dd e\ \dd M_2\ \dd v_{\rm sys} \nonumber \\
& & \qquad \times P( P_{\rm orb}, e, M_2, v_{\rm sys}, D \given \vec{x}_{\rm i}, M).
\end{eqnarray}
We substitute for $D$, and based on independence we factor the integrand into separate, tractable parts:
\begin{eqnarray}
P(D \given \vec{x}_{\rm i}, M) &=&  \int \dd P_{\rm orb}\ \dd e\ \dd M_2\ \dd v_{\rm sys}\ P(P'_{\rm orb} \given P_{\rm orb}) \nonumber \\
	& & \qquad \times P(e' \given e)\ P(M'_2 \given M_2) \nonumber \\
	& & \qquad \times P(P_{\rm orb}, e, M_2, v_{\rm sys} \given \vec{x}_{\rm i}) \nonumber \\
	& & \qquad \times P(\alpha, \delta \given \vec{x}_{\rm i}, v_{\rm sys}). \label{eq:marginalized}
\end{eqnarray}


The first three terms in the integrand of Equation \ref{eq:marginalized}, $P(P'_{\rm orb} \given P_{\rm orb})$, $P(e' \given e)$, and $P(M'_2 \given M_2)$, account for the observational uncertainties on the binary's orbit. We model these with Gaussian uncertainties. In the future, $P(M'_2 \given M_2)$ could be based on the photometric data for the donor star, which in combination with stellar evolution models, can constrain its mass. We discuss these three Gaussian uncertainties along with the fourth term in the integrand, which describes the function evolving the binary from its {\it ab initio} state to the $P_{\rm orb}$, $e$, $M_2$, and $v_{\rm sys}$ of the system today, in Section \ref{sec:likelihood_ind_binary}.

The last term in the integrand accounts for the fact that the system's birth place will, in general, be different from its observed position since the centre of mass of a system received a kick during the primary's core collapse. We explicitly include the dependence on $\vec{x_{\rm i}}$ and $v_{\rm sys}$ since the distance travelled depends on both the system's velocity and the time since the primary's SN. We derive this term in Section \ref{sec:ra_dec} below.




% Priors
\subsection{Prior Probabilities: $P(\vec{x}_{\rm i})$} 
\label{sec:priors}



Typically prior probabilities are determined within rapid BPS codes. In our model, BPS codes are used only to rapidly evolve binaries from their initial conditions. Since we define the prior distributions within our model, we describe them here. Our model includes between eight and 12 parameters, which can be factored into several parts:
\begin{eqnarray}
P(\vec{x}_{\rm i}) &=& P(M_{\rm 1,i}) P(M_{\rm 2,i}\given M_{\rm 1,i}) P(a_{\rm i}) \nonumber \\
 & & \qquad  \times\ P(e_{\rm i}) P(\vec{v}_{\rm k,1}) P(\vec{v}_{\rm k,2}) P(\alpha_{\rm i}, \delta_{\rm i}, t_{\rm i})
\end{eqnarray}
We discuss each of these terms in turn below.




\subsubsection{Initial Binary Parameters}

Our prior probabilities over $M_{\rm 1,i}$, $M_{\rm 2,i}$, $a_{\rm i}$, and $e_{\rm i}$ are all equivalent or similar to the distributions used in other population synthesis codes \citep[e.g.,][]{belczynski08}.

The initial primary mass follows a power law:
\begin{equation}
P(M_{\rm 1,i}) = C_{\rm m} M_{\rm 1,i}^{\alpha};\ M_{\rm 1,i} \in [M_{\rm 1,min}, M_{\rm 1,max}]
%[8\Msun,\ 30\Msun],
\end{equation}
where $C_{\rm m}$ is a normalization constant dependent upon the limits of the distribution ($M_{\rm 1,min} = 8 \Msun$ and $M_{\rm 1,max} = 39 \Msun$) and $\alpha$:
\begin{equation}
C_{\rm m} = \frac{\alpha + 1}{M_{\rm 1,max}^{\alpha+1} - M_{\rm 1,min}^{\alpha+1}}.
\end{equation}
We choose 8 and 30 \Msun\ as the lower and upper mass limits that produce a NS; we do not consider HMXBs formed with BH accretors. In the present analysis, since the distribution strongly preferences lower mass stars, our results are relatively independent of the upper mass limit. We choose a Salpeter power law: $\alpha = -2.35$. 

We choose a prior on the secondary mass based on a flat mass-ratio distribution which has the subtle effect that the prior on the secondary is dependent on the primary. The maximum mass-ratio is unity to ensure the primary is the more massive of the pair, and the minimum mass is set to 2 \Msun. 
%informed by the limit for stable, thermal-time-scale mass transfer. Polytropic stellar models indicate that mass transfer will always be dynamically unstable for a more massive donor with a convective envelope \citep{hjellming87}. Such systems would evolve through evolutionary channels not considered here. Recent work using more realistic stellar models by \citet{ge15} and \citet{pavlovskii15} has cast doubt on this conclusion, suggesting instead that mass transfer may be stable for binaries with lower mass ratios. Lacking a comprehensive treatment of the lower mass ratio limit for stable mass transfer for donors with convective envelopes, we treat stars with convective and radiative envelopes equivalently and set the lower mass ratio limit to 0.3, the same as used by \citet{bhadkamkar12}. We note that this choice is also in rough agreement with the findings by \citet{pavlovskii15}. 
This leads to a prior probability:
\begin{equation}
P(M_{\rm 2,i} \given M_{\rm 1,i}) = \frac{1}{0.7 M_{\rm 1,i}};\ M_{\rm 2,i} \in [0.3 M_{\rm 1,i}, 1.0 M_{\rm 1,i}]
\end{equation}

We choose a prior on the initial orbital separation of the binary that scales with $a_{\rm i}^{-1}$ \citep{abt83}:
\begin{equation}
P(a_i \given M_{\rm 1,i}, M_{\rm 2,i}, e_{\rm i}) = \frac{C_{\rm a}}{a_{\rm i}};\ a_{\rm i} \in [a_{\rm min}, a_{\rm max}],
\end{equation}
where $C_{\rm a}$ is a normalization constant
\begin{equation}
C_{\rm a} = \frac{1}{\log a_{\rm max} - \log a_{\rm min}},
\end{equation}
and $a_{\rm min}$ is the greater of either 10 $R_{\odot}$ or the separation such that the primary will not overfill its Roche lobe at ZAMS, and $a_{\rm max} = 10^4 R_{\odot}$. 

Finally, we choose a thermal initial eccentricity distribution \citep{duquennoy91}:
\begin{equation}
P(e_{\rm i}) = 2e_{\rm i};\ e_{\rm i} \in[0,1].
\end{equation}




\begin{figure*}
\begin{center}
\includegraphics[width=0.95\textwidth]{fig1.pdf}
\caption{The prior on both position of the binary's birth location and time depends on the SMC star-formation history maps derived from \citet{harris04}. We show samples of the star-formation history at four different times spanning the range of typical HMXB lifetimes. These demonstrate the typical resolution of the spatially resolved star-formation history.}
\label{fig:SMC_SFH}
\end{center}
\end{figure*}


\subsubsection{SN Kick Parameters}

The SN kick velocity, $\vec{v}_{\rm k}$, is composed of three parameters, which we model as a kick magnitude ($v_{\rm k}$) and two angles ($\theta_{\rm k}, \phi_{\rm k}$). If the binary is a double compact object, the two SNe are independent. Typically, these parameters are determined through Monte Carlo random draws from a specific distribution on-the-fly during each binary's evolution. In our model, we instead include the SN kick magnitude and direction as model parameters with prior probabilities corresponding to standard distributions; we assume that $v_{\rm k}$ follows a Maxwellian distribution with a dispersion of 265 km s$^{-1}$ \citep{hobbs05}. We can therefore express the normalized probability of $v_{\rm k}$ as:
\begin{equation}
P(v_{\rm k}) = \sqrt{\frac{2}{\pi}} \frac{v_{\rm k}^2} {\sigma^3} {\rm exp} \left[ -v_{\rm k}^2 / 2 \sigma^2 \right];\ v_{\rm k} \in [0, \infty). \label{eq:P_v_k}
\end{equation}


Since the kick distribution is assumed to be isotropic, normalized probabilities for the kick polar, $\theta_{\rm k}$, and azimuthal, $\phi_{\rm k}$, angles are straightforward:
\begin{eqnarray}
P(\theta_{\rm k}) &=& \frac{\sin \theta_{\rm k}}{2};\ \theta_{\rm k} \in [0, \pi] \label{eq:P_theta_k} \\
P(\phi_{\rm k}) &=& \frac{1}{\pi};\ \phi_{\rm k} \in [0, \pi] . \label{eq:P_phi_k}
\end{eqnarray}
Formally, $\phi_{\rm k}$ should vary between 0 and $2 \pi$; however, as described in Section \ref{sec:trans_SN}, the only contribution of $\phi_{\rm k}$ to the evolution of the binary is through a $\sin^2 \phi_{\rm k}$ term, which is periodic from 0 to $\pi$.





\subsubsection{Star-Formation History}


The priors on $\alpha_{\rm i}$, $\delta_{\rm i}$, and $t_{\rm i}$ depend on the local star-formation history at that position and time. Our model can be run with a basic time dependent star formation history, ignoring the systems' positions, but the most power is gained by including spatially resolved star-formation history maps as a prior on both position and time. We use the SMC star-formation history maps from \citet{harris04}. These maps cover the SMC with $\sim$1300 separate regions with angular resolutions of 12\amin\ on a side in the inner regions and 24\amin\ on a side in the outer regions. The star-formation history for each region has a resolution of 0.2 dex in $t$ ranging from 6.8 to 10.2 in log $t$. We ignore uncertainties on the star-formation histories, generating linear interpolation functions for each of the 1300 regions. We only take into account star-formation at a metallicity of $Z=0.008$, the dominant metallicity at which stars have been formed in the SMC over the past 1 Gyr. Since HMXBs have been born in the past 10$^8$ yrs, the older, lower-$Z$ population is irrelevant for the HMXB we study in this work.

These histories provide ${\rm SFR}(\alpha_{\rm i}, \delta_{\rm i}, t_{\rm i})$, the rate per area on the sky that stars were formed at a specific location and time in the SMC. With a normalization constant this spatially dependent star-formation rate is the prior on position and time:
\begin{equation}
P(\alpha_{\rm i}, \delta_{\rm i}, t_{\rm i}) = \frac{1}{N_{\rm SMC}} {\rm SFR}(\alpha_{\rm i}, \delta_{\rm i}, t_{\rm i}),
\end{equation}
where $N_{\rm SMC}$ is the number of stars with $Z=0.008$ produced throughout the lifetime of the SMC. Figure \ref{fig:SMC_SFH} shows the star-formation rate map for four different times spanning the range of typical HMXB lifetimes. These maps indicate that the locations of high star-formation regions have substantially evolved over the past 5$\times10^7$ years.



% \subsection{Priors} \label{sec:priors_indiv}


% \begin{figure}
% \begin{center}
% \includegraphics[width=0.95\columnwidth]{fig2.pdf}
% \caption{The prior on both position of the binary's birth location and time depends on the star-formation history. Only the star-formation history within the cone, the size and shape of which is set by the binary parameters, are taken into account. We show contours indicating regions of high (red) and low (blue) star-formation for three different ages. Determining the prior on the star-formation history involves integrating the star-formation history throughout the cone.}
% \label{fig:prior_SFH}
% \end{center}
% \end{figure}


% The binary evolution prescriptions and SN kick priors are the same as described in Section \ref{sec:priors}. Only the star-formation rate {\it normalization} for the prior on $\alpha_{\rm i}$, $\delta_{\rm i}$, and $t_{\rm i}$ differs; Figure \ref{fig:prior_SFH} demonstrates the source of the difference.


% An object at a current location $(\alpha, \delta)$ could have been formed only within a region (bounded by circles) that progressively increases for older birth times since older systems have had more time to travel from their birth locations. Only stars formed within this cone contribute to the normalization constant; stars formed outside the cone could not have led to the observed system. The shape and size of the cone changes depending on the binary parameters (the `slope' of the cone depends on $v_{\rm sys}$), therefore the prior needs to be recalculated for each set of model parameters. Because of the function calls, determining this normalization constant is the most computationally expensive calculation of this method. 

% The normalization constant is determined by integrating over the spatially resolved star-formation rate throughout the cone and setting the quantity to unity:
% \begin{equation}
% 1 = C_{\rm SFH}\ \int_{t_{\rm min}}^{t_{\rm max}} \int_0^{2 \pi} \int_0^{\theta_{\rm C}} \dd t_{\rm i}\ \dd \phi\ \dd \theta\ {\rm SFR}(\theta, \phi, t_{\rm i}). 
% \end{equation}
% where $\theta$ and $\phi$ are the polar and azimuthal angles (separate from similar parameters used to describe the SN kick direction) defining the cone. See Figure \ref{fig:prior_SFH} for the definition of these angles. $\theta_{\rm C}$ is the critical angle bounding the cone:
% \begin{equation}
% \theta_{\rm C} = \frac{v_{\rm sys} \left[ t_{\rm i} - t(M_{\rm 1,i}) \right]}{D_{\rm SMC}}, \label{eq:theta_c}
% \end{equation}
% where $t(M)$ is the lifetime of a star of mass $M$ and the distance to the SMC, $D_{\rm SMC}$, is 61 kpc \citep{hilditch05}. The bounds on the integral over $t_{\rm i}$ are set by the minimum and maximum times during which the system can emit X-rays: 
% \begin{eqnarray}
% t_{\rm min} &=& t(M_{\rm 1,i}) \\
% t_{\rm max} &=& t(M_{\rm 2,MT}) - t_{\rm eff} + t_{\rm min},
% \end{eqnarray}
% where $t_{\rm eff}$ is defined in Equation \ref{eq:t_eff_1}.

% We choose to calculate the integral using a Monte Carlo sampling technique. We draw $N$ random samples throughout the cone. The integral is the product of the average of the samples and the cone's volume:
% \begin{equation}
% \frac{1}{C_{\rm SFH}} \approx \frac{\pi \theta_{\rm C}^2 (t_{\rm max} - t_{\rm min})}{3N} \sum_j {\rm SFR}(\theta_j, \phi_j, t_{{\rm i},j}),
% \end{equation}
% where $(\theta_j, \phi_j, t_{{\rm i},j})$ are random samples, evenly distributed around the cone's volume. Formally, an infinite number of samples is required for the approximation above to become an equality. However, we find that for typical systems, 512 random samples provide a sufficiently accurate estimate of $C_{\rm SFH}$.



% Random samples of $\phi$ are drawn from a uniform distribution between zero and $2\pi$. To obtain random samples of $\theta$ and $t$, we use inversion sampling, which involves obtaining random samples of the cumulative distribution function. The random values of $\theta$ and $t$ are the inversions of those samples:
%  \begin{eqnarray}
% \phi &\sim& U(0, 2\pi) \\
% \theta &=& \sqrt{y_{\theta}} \theta_{\rm C}: y_{\theta} \sim U(0,1) \\
% t &=& \sqrt[3]{y_{t}}\left( t_{\rm max} - t_{\rm min} \right) + t_{\rm min}: y_t \sim U(0,1).
% \end{eqnarray}



\subsection{Binary Parameter Likelihood} \label{sec:likelihoods}

Given a birth time and a particular set of initial binary parameters, the binary likelihood provides the probability that a binary of interest will be formed. For populations of systems, the likelihood is simply the function provided in Equation \ref{eq:}. Determining this function nevertheless requires evolving the system through its evolution. Currently, this requires rapid binary evolution codes. We provide a modified version of one such code, \bse\ along with \dart. Our minor modifications are described in Appendix \ref{sec:pybse}

\subsubsection{Fitting Observables with Uncertainties}

If we would like to model an individual system with a set of observations, the likelihood function includes the observations and their uncertainties. For evolved stellar binaries, determining this term relies first on the evolution from $\vec{x}_{\rm i}$ to $\vec{x}_{\rm f}$ described determined by our rapid binary evolution code provides the term $P(P_{\rm orb}, e, M_2, v_{\rm sys} \given \vec{x}_{\rm i})$ in Equation \ref{eq:marginalized}. 

Then we need to compare with observations of the system. For an observed parameter with known, Gaussian standard deviation (measurement uncertainty), the likelihood function is straightforward and involves the evaluation of the probability density of the normalized Gaussian distribution at $\vec{x}_{\rm f}$:
\begin{equation}
P(x'|x) = \mathcal{N}(x' | x, \sigma_{x'}).
\end{equation}

Our method can adapt to any observation of an individual stellar parameter included in our model; we only need to compare the evolved binary parameters to the observations. To simultaneously fit multiple observations, the resulting likelihood is then a product of the fits to each of the individual observed quantities:
\begin{equation}
P(\vec{x}' | \vec{x}_{\rm f}) = \prod_k \mathcal{N}(x'_k | x_k, \sigma_{x'_k}),
\end{equation}
for all $k$ observed quantities.


\subsubsection{Mass Function}
\label{sec:mass_function}





\subsubsection{Position Likelihood} 
\label{sec:ra_dec}

The position likelihood provides the probability that, given a system's position, systemic velocity, and time since SN, the system would be observed at its current position. 
% Since HMXBs are generally short lived, this probability is non-zero for only a small region around any given HMXB's observed position. 
To solve the positional component of Equation \ref{eq:marginalized}, we first marginalize over $\omega$, the angle between the line of sight vector to the birth location and the systemic velocity vector:
\begin{equation}
P(\alpha, \delta \given \vec{x}_{\rm i}, v_{\rm sys} ) = \int \dd \omega\ P(\alpha, \delta, \omega \given \vec{x}_{\rm i}, v_{\rm sys} ). \label{eq:P_pos_1}
\end{equation}

\begin{figure}
\begin{center}
\includegraphics[width=0.95\columnwidth]{fig3.pdf}
\caption{Our representation of the current position $(\alpha, \delta)$ in relation to its birth position $(\alpha_{\rm i}, \delta_{\rm i})$. The distance the system traveled is $d$, which has a projected separation $s$. We express this transportation as a function of $\theta_{\rm proj}$ and position angle, $\phi$. Note, for typical systems $d << D_{\rm SMC}$.}
\label{fig:position_projection}
\end{center}
\end{figure}


We next perform a coordinate transformation from the absolute positional coordinates $\alpha$ and $\delta$ to the relative angular separation, $\theta_{\rm proj}$, and the position angle, $\phi$, measured from the system's birth location. Figure \ref{fig:position_projection} shows our parameterization of the transformation from a system's birth location at $\alpha_{\rm i}$ and $\delta_{\rm i}$ to its current location at $\alpha$ and $\delta$. The determinant of the Jacobian matrix for this transformation is:
\begin{equation}
J_{\rm coor} = \left| \frac{\partial \theta_{\rm proj}}{\partial \alpha} \frac{\partial \phi}{\partial \delta} - \frac{\partial \theta_{\rm proj}}{\partial \delta}\frac{\partial \phi}{\partial \alpha}  \right|.
\end{equation}
These partial derivatives can be calculated by taking the derivatives of standard formulae for the angular separation and position angle of double stars. Equation \ref{eq:P_pos_1} now becomes:
\begin{eqnarray}
P(\alpha, \delta \given \vec{x}_{\rm i}, v_{\rm sys} ) &=& \int \dd \omega\ P(\theta_{\rm proj}, \phi, \omega \given \vec{x}_{\rm i}, v_{\rm sys} )\ J_{\rm coor} \nonumber \\
&=& \int \dd \omega\ P(\theta_{\rm proj} \given \omega,  \vec{x}_{\rm i}, v_{\rm sys} ) \nonumber \\
& & \qquad \times\ P(\phi)\ P(\omega)\ J_{\rm coor}, \label{eq:P_pos_2}
\end{eqnarray}
where we have separated terms based on independence. $\omega$ is a randomly chosen polar angle and $\phi$ is a randomly chosen azimuthal angle: 
\begin{eqnarray}
P(\omega) &=& \frac{\sin \omega} {2};\ \omega \in [0,\pi] \\
P(\phi) &=& \frac{1}{2 \pi};\ \phi \in [0, 2\pi].
\end{eqnarray}


The physical distance a system travels is the product of $v_{\rm sys}$ and the time since the primary's core collapse:
\begin{equation}
d = v_{\rm sys} \left[t_{\rm i} - t(M_{\rm 1,i}) \right].
\end{equation}
We ignore the effects of the SMC's gravitational potential and assume the systems move in free space after receiving a kick. We can only observe the projection of $d$ onto the sky, $s = d \sin \omega$. We can also approximate $s$ as the product of $D_{\rm SMC}$ and $\theta_{\rm proj}$. After equating these two expressions for $s$ and solving for $\theta_{\rm proj}$, the first term of the integrand in Equation \ref{eq:P_pos_2} becomes a delta function:
\begin{equation}
P(\theta_{\rm proj} \given \omega, \vec{x}_{\rm i}, v_{\rm sys}) = \delta \left[G(\omega)\right], \label{eq:P_theta_proj}
\end{equation}
where:
\begin{equation}
G(\omega) = \theta_{\rm proj} - \frac{v_{\rm sys} \left[t_{\rm i} - t(M_{\rm 1,i}) \right] \sin \omega}{D_{\rm SMC}}.\end{equation}


With the delta function from Equation \ref{eq:P_theta_proj}, the integral in Equation \ref{eq:P_pos_2} can be reduced:
\begin{equation}
\int \dd \omega\ P(\phi) P(\omega) \delta \left[ G(\omega) \right]  J_{\rm coor}\  =\ \sum_j\ \frac{P(\omega_j^{\star}) P(\phi)  J_{\rm coor}}{ \left| \frac{ \dd G (\omega) }{\dd \omega} \right|_{\omega_j^*}},
\end{equation}
where the sum is over $\omega_j^*$, the $j$ roots of $G(\omega)$. There are two roots corresponding to whether the object is in front of or behind its birth location. This integral can now be evaluated analytically:
\begin{equation}
P(\alpha, \delta \given \vec{x}_{\rm i}, v_{\rm sys} ) =
\begin{cases} 
      0, & \theta_{\rm proj} \geq \theta_{\rm C}\\
     \frac{\tan \omega^*}{2 \pi \theta_{\rm C}}  J_{\rm coor}, & \theta_{\rm proj} < \theta_{\rm C} 
   \end{cases}
\end{equation}
where:
\begin{equation}
\omega^{\star} = \sin^{-1} \left[ \frac{\theta_{\rm proj}}{\theta_{\rm C}} \right].
\end{equation}



























\begin{comment}



\section{Statistical Method} \label{sec:stats}


BPS randomly produces a set of binaries by evolving the distribution of initial binary parameters through binary evolution prescriptions. To first order, high mass binaries can be determined uniquely by only a few parameters: the binary's initial masses, $M_{\rm 1,i}$ and $M_{\rm 2,i}$, its separation, $a_{\rm i}$, and eccentricity, $e_{\rm i}$, the kick velocity it received after the primary collapsed to form a NS or BH, $\vec{v}_{\rm k}$, and its birth time $t_{\rm i}$. Depending on the population being modeled, we may optionally include the coordinates for the binary's birth position, $\alpha_{\rm i}$ and $\delta_{\rm i}$, and the kick velocity received when the secondary collapsed to form a compact object:
\begin{equation}
\vec{x}_{\rm i} \equiv ( M_{\rm 1,i}, M_{\rm 2,i}, a_{\rm i}, e_{\rm i}, \vec{v}_{\rm k, SN1}, \vec{v}_{\rm k, SN2}, \alpha_{\rm i}, \delta_{\rm i}, t_{\rm i} ). \label{eq:x_i}
\end{equation}


From a Bayesian perspective, we would like to determine the set of binaries simulated by a particular binary evolution model: $P(\vec{x}_{\rm i} \given M)$. For clarity, we will drop the $M$ throughout the remainder of this analysis. Typically, we are interested in selecting those systems that appear similar to one particular observed system or a population of systems (e.g., HMXBs with X-ray luminosities above some detection limit). We can denote this probability mathematically as $P(\vec{x}_{\rm i} \given D)$, where $D$ describes the constraint on the resulting systems of interest, whether data or abstract constraints. Using Bayes's theorem, this can be described:
\begin{equation}
P(\vec{x}_{\rm i} \given D) = \frac{P(D \given \vec{x}_{\rm i})\ P(\vec{x}_{\rm i})}{P(D)}, \label{eq:bayes_ind}
\end{equation}
where $P(D \given \vec{x}_{\rm i})$ is the likelihood function, $P(\vec{x}_{\rm i})$ is the prior probability on the model parameters, and $P(D)$ is a normalization constant that for our purposes can be ignored.


Traditional BPS uses an intuitive, but brute force, method to identify $P(\vec{x}_{\rm i} \given D)$ that consists of three parts: First, generate a random distribution of $\vec{x}_{\rm i}$ based on observationally-motivated prior probabilities for initial binaries:
\begin{equation}
\vec{x}_{\rm i} \sim P(\vec{x}_{\rm i}).
\end{equation}
Second, use a rapid binary evolution code to evolve those binaries from their initial state ($\vec{x}_{\rm i}$) until its current state ($\vec{x}_{\rm f}$):
\begin{equation}
\vec{x}_{\rm f} = f(\vec{x}_{\rm i}). \label{eq:xf_xi}
\end{equation}
We note here that although we describe $\vec{x}_{\rm f}$ as a set of parameters describing the current state of the binary, it may also include information about the binary at any point along its evolution. Finally, select the subset of those modeled binaries that are similar to the observations. Results are then derived from the selected subset of systems:
\begin{equation}
P(D | \vec{x}_{\rm f}).
\end{equation}

For example, if one wants to determine the orbital period and eccentricity distribution of double NSs \citep{andrews15}, one evolves a population of high mass binaries and the orbital period and eccentricity distribution from the resulting systems that exist as DNSs. Mathematically:
\begin{equation}
P(P_{\rm orb}, e | x_{\rm HMXB}) \approx \frac{1}{N} \sum_i^N f(\vec{x}_{\rm i}),
\end{equation}
where $f$ is a function providing the $P_{\rm orb}$ and $e$ calculated by a binary evolution code for a set of initial conditions, $\vec{x}_{\rm i}$.

For example, since only a small subset of the binaries that form HMXBs will be consistent with any particular observed system, the likelihood function may be non-zero for only a small region of parameter space. The more precisely a system is measured, the smaller the phase space volume of interest.



Since in general we do not {\it a priori} know which initial conditions will produce binaries which we are interested in modeling (for instance, many systems are disrupted or merge during their evolution), we must test the entire region of initial binary parameter space that could plausibly produce them. 


Traditional BPS takes a shotgun approach, making many random draws of $\vec{x}_{\rm i}$ from $P(\vec{x}_{\rm i})$. 

 

One key aspect of this parameterization is that the SN kick magnitude and direction is included as model parameters rather than determined from random draws on-the-fly during binary evolution.




If we would like to quantitatively compare a model to a set of observationally derived properties, $D$, of a particular system rather than a population, we need to adapt our method. We may be interested in either deriving the initial binary conditions that could have produced the observed systems, $P(\vec{x}_{\rm i} \given D)$, or determining the current binary parameters, $P(\vec{x}_{\rm f} \given D)$. These two quantities are closely related since binary evolution directly relates $\vec{x}_{\rm i}$ to $\vec{x}_{\rm f}$. 












We simulate individual HMXBs using the same model parameters that we use for a population of HMXBs, so $\vec{x}_{\rm i}$ is determined by Equation \ref{eq:x_i}. The priors on these parameters, which we discuss in Section \ref{sec:priors_indiv}, are somewhat different for individual systems, compared with those derived for populations of HMXBs.

Individual HMXBs in the SMC may have well measured $P'_{\rm orb}$, $e'$, and $M'_2$ (primed quantities indicate observed rather than true, underlying quantities) determined from X-ray timing and observations of the optical counterpart to the X-ray source. Each of these measured quantities has some uncertainty associated with it which should be taken into account. Furthermore, individual HMXBs have a specific location that we are trying to associate with nearby star forming regions. We start by defining $D$ as:
\begin{equation}
D \equiv (\alpha, \delta, P'_{\rm orb}, e', M'_2). \label{eq:D}
\end{equation}
Uncertainties on the primed quantities are not explicitly included in $D$, and we ignore uncertainties on the position.









This allows us to re-envision the formation of binary systems in a somewhat different manner. Stellar binaries are formed out of then entire range of binary parameter space, at a rate determined by the multidimensional prior probability distribution for that space. However, only a (potentially small) region of that space produces the binaries of interest. The likelihood function determines the boundaries of that space.



For binary populations involving NSs and black holes (BH), such as HMXBs, traditional BPS may be an inefficient tool; mass transfer and SN kicks may merge or disrupt the majority of systems before they evolve into objects of interest (i.e., $P(x_{\rm HMXB} \given \vec{x}_{\rm i}) = 0$ for many or even most of the randomly drawn $\vec{x}_{\rm i}$). The fact that these binaries are discarded is a major source of computational expense in traditional BPS. 







% MCMC population synthesis
Instead of random draws of $\vec{x}_{\rm i}$, we consider the components of $\vec{x}_{\rm i}$ model parameters. In this formulation $P(\vec{x}_{\rm i})$ is the prior probability on the model parameters, and $P(x_{\rm type} \given \vec{x}_{\rm i})$ is the likelihood of producing a binary of a particular type from a given $\vec{x}_{\rm i}$. Using Bayes' Theorem, we can then identify the set of $\vec{x}_{\rm i}$ most likely to produce these binaries:
\begin{equation}
P(\vec{x}_{\rm i} \given x_{\rm type}) = \frac{P(x_{\rm type} \given \vec{x}_{\rm i}) P(\vec{x}_{\rm i})} {P(x_{\rm type})}. \label{eq:bayes_pop}
\end{equation}
We ignore $P(x_{\rm type})$, which serves as a normalization constant, and define the posterior probability as the numerator on the right hand side of Equation \ref{eq:bayes_pop}. Although in this work we ignore the denominator, it could be used to account for observational biases in future studies. The large dimensionality of $\vec{x}_{\rm i}$ argues for an efficient numerical method to probe the region of viable parameter space.


In an MCMC code, a `walker' moves around the $\vec{x}_{\rm i}$ parameter space: the posterior probability of the current $\vec{x}_{\rm i}$ is calculated, a new trial $\vec{x}_{\rm i}$ is randomly selected, the posterior probability of the new position is compared to that of the current position, and depending on the ratio of the two posterior probabilities, the new $\vec{x}_{\rm i}$ is either selected and added to the chain or rejected and the current position is kept for another step. The chain stores a record of all the walker's past positions. Samples from this chain comprise the synthetic population analogous to the population generated by traditional BPS.\footnote{Since the current walker position will necessarily be closely related to the previous step, the posterior samples will be correlated with some characteristic length. The autocorrelation length needs to be calculated {\it a posteriori}, and only one sample per autocorrelation length can be considered independent. We return to this in Section \ref{sec:limitations}.}


In principle, for an infinite number of iterations, the distribution of posterior samples of $\vec{x}_{\rm i}$ produced by this method will identically mimic the distribution generated by traditional BPS. Since we are, unfortunately, limited to a finite sample, the computation time to produce a statistically robust sample using each method depends on the relative formation efficiency of HMXBs and the autocorrelation length of the MCMC posterior distribution. For high formation efficiencies, traditional methods are preferred, since every random draw from BPS is independent. However for high mass binaries with low formation efficiencies, two separate effects make MCMC the preferred choice: most high mass binaries disrupt, merger, or never interact, and those that do typically only exist in an interesting phase of their evolution for a short portion of their lifetime.


Correct implementation requires careful attention to the prior distributions, $P(\vec{x}_{\rm i})$, and an efficient method to calculate $P(x_{\rm type} \given \vec{x}_{\rm i})$. We describe how we calculate the prior probabilities in Section \ref{sec:priors} and the likelihoods in Section \ref{sec:likelihoods}.




% Priors
\subsection{Prior Probabilities: $P(\vec{x}_{\rm i})$} \label{sec:priors}

Typically prior probabilities are determined within rapid BPS codes. In our model, BPS codes are used only to rapidly evolve binaries from their initial conditions. Since we define the prior distributions within our model, we describe them here. Our model includes ten parameters, which can be factored into several parts:
\begin{eqnarray}
P(\vec{x}_{\rm i}) &=& P(M_{\rm 1,i}) P(M_{\rm 2,i}\given M_{\rm 1,i}) P(a_{\rm i}) \nonumber \\
 & & \qquad  \times\ P(e_{\rm i}) P(\vec{v}_{\rm k,1}) P(\vec{v}_{\rm k,2}) P(\alpha_{\rm i}, \delta_{\rm i}, t_{\rm i})
\end{eqnarray}
We discuss each of these terms in turn below.

\subsubsection{Initial Binary Parameters}

Our prior probabilities over $M_{\rm 1,i}$, $M_{\rm 2,i}$, $a_{\rm i}$, and $e_{\rm i}$ are all equivalent or similar to the distributions used in other population synthesis codes \citep[e.g.,][]{belczynski08}.

The initial primary mass follows a power law:
\begin{equation}
P(M_{\rm 1,i}) = C_{\rm m} M_{\rm 1,i}^{\alpha};\ M_{\rm 1,i} \in [M_{\rm 1,min}, M_{\rm 1,max}]
%[8\Msun,\ 30\Msun],
\end{equation}
where $C_{\rm m}$ is a normalization constant dependent upon the limits of the distribution ($M_{\rm 1,min} = 8 \Msun$ and $M_{\rm 1,max} = 39 \Msun$) and $\alpha$:
\begin{equation}
C_{\rm m} = \frac{\alpha + 1}{M_{\rm 1,max}^{\alpha+1} - M_{\rm 1,min}^{\alpha+1}}.
\end{equation}
We choose 8 and 30 \Msun\ as the lower and upper mass limits that produce a NS; we do not consider HMXBs formed with BH accretors. In the present analysis, since the distribution strongly preferences lower mass stars, our results are relatively independent of the upper mass limit. We choose a Salpeter power law: $\alpha = -2.35$. 

We choose a prior on the secondary mass based on a flat mass-ratio distribution which has the subtle effect that the prior on the secondary is dependent on the primary. The maximum mass-ratio is unity to ensure the primary is the more massive of the pair, and the minimum mass is set to 2 \Msun. 
%informed by the limit for stable, thermal-time-scale mass transfer. Polytropic stellar models indicate that mass transfer will always be dynamically unstable for a more massive donor with a convective envelope \citep{hjellming87}. Such systems would evolve through evolutionary channels not considered here. Recent work using more realistic stellar models by \citet{ge15} and \citet{pavlovskii15} has cast doubt on this conclusion, suggesting instead that mass transfer may be stable for binaries with lower mass ratios. Lacking a comprehensive treatment of the lower mass ratio limit for stable mass transfer for donors with convective envelopes, we treat stars with convective and radiative envelopes equivalently and set the lower mass ratio limit to 0.3, the same as used by \citet{bhadkamkar12}. We note that this choice is also in rough agreement with the findings by \citet{pavlovskii15}. 
This leads to a prior probability:
\begin{equation}
P(M_{\rm 2,i} \given M_{\rm 1,i}) = \frac{1}{0.7 M_{\rm 1,i}};\ M_{\rm 2,i} \in [0.3 M_{\rm 1,i}, 1.0 M_{\rm 1,i}]
\end{equation}

We choose a prior on the initial orbital separation of the binary that scales with $a_{\rm i}^{-1}$ \citep{abt83}:
\begin{equation}
P(a_i \given M_{\rm 1,i}, M_{\rm 2,i}, e_{\rm i}) = \frac{C_{\rm a}}{a_{\rm i}};\ a_{\rm i} \in [a_{\rm min}, a_{\rm max}],
\end{equation}
where $C_{\rm a}$ is a normalization constant
\begin{equation}
C_{\rm a} = \frac{1}{\log a_{\rm max} - \log a_{\rm min}},
\end{equation}
and $a_{\rm min}$ is the greater of either 10 $R_{\odot}$ or the separation such that the primary will not overfill its Roche lobe at ZAMS, and $a_{\rm max} = 10^4 R_{\odot}$. 

Finally, we choose a thermal initial eccentricity distribution \citep{duquennoy91}:
\begin{equation}
P(e_{\rm i}) = 2e_{\rm i};\ e_{\rm i} \in[0,1].
\end{equation}



\subsubsection{SN Kick Parameters}

The SN kick velocity, $\vec{v}_{\rm k}$, is composed of three parameters, which we model as a kick magnitude ($v_{\rm k}$) and two angles ($\theta_{\rm k}, \phi_{\rm k}$). If the binary is a double compact object, the two SNe are independent. Typically, these parameters are determined through Monte Carlo random draws from a specific distribution on-the-fly during each binary's evolution. In our model, we instead include the SN kick magnitude and direction as model parameters with prior probabilities corresponding to standard distributions; we assume that $v_{\rm k}$ follows a Maxwellian distribution with a dispersion of 265 km s$^{-1}$ \citep{hobbs05}. We can therefore express the normalized probability of $v_{\rm k}$ as:
\begin{equation}
P(v_{\rm k}) = \sqrt{\frac{2}{\pi}} \frac{v_{\rm k}^2} {\sigma^3} {\rm exp} \left[ -v_{\rm k}^2 / 2 \sigma^2 \right];\ v_{\rm k} \in [0, \infty). \label{eq:P_v_k}
\end{equation}


\begin{figure*}
\begin{center}
\includegraphics[width=0.95\textwidth]{fig1.pdf}
\caption{The prior on both position of the binary's birth location and time depends on the SMC star-formation history maps derived from \citet{harris04}. We show samples of the star-formation history at four different times spanning the range of typical HMXB lifetimes. These demonstrate the typical resolution of the spatially resolved star-formation history.}
\label{fig:SMC_SFH}
\end{center}
\end{figure*}



Since the kick distribution is assumed to be isotropic, normalized probabilities for the kick polar, $\theta_{\rm k}$, and azimuthal, $\phi_{\rm k}$, angles are straightforward:
\begin{eqnarray}
P(\theta_{\rm k}) &=& \frac{\sin \theta_{\rm k}}{2};\ \theta_{\rm k} \in [0, \pi] \label{eq:P_theta_k} \\
P(\phi_{\rm k}) &=& \frac{1}{\pi};\ \phi_{\rm k} \in [0, \pi] . \label{eq:P_phi_k}
\end{eqnarray}
Formally, $\phi_{\rm k}$ should vary between 0 and $2 \pi$; however, as described in Section \ref{sec:trans_SN}, the only contribution of $\phi_{\rm k}$ to the evolution of the binary is through a $\sin^2 \phi_{\rm k}$ term, which is periodic from 0 to $\pi$.


\subsubsection{Star-Formation History}


The priors on $\alpha_{\rm i}$, $\delta_{\rm i}$, and $t_{\rm i}$ depend on the local star-formation history at that position and time. We use the SMC star-formation history maps from \citet{harris04}. These star-formation history maps cover the SMC with $\sim$1300 separate regions with angular resolutions of 12\amin\ on a side in the inner regions and 24\amin\ on a side in the outer regions. The star-formation history for each region has a resolution of 0.2 dex in $t$ ranging from 6.8 to 10.2 in log $t$. We ignore uncertainties on the star-formation histories, generating linear interpolation functions for each of the 1300 regions. We only take into account star-formation at a metallicity of $Z=0.008$, the dominant metallicity at which stars have been formed in the SMC over the past 1 Gyr. Since HMXBs have been born in the past 10$^8$ yrs, the older, lower-$Z$ population is irrelevant for the HMXB we study in this work.

These histories provide ${\rm SFR}(\alpha_{\rm i}, \delta_{\rm i}, t_{\rm i})$, the rate per area on the sky that stars were formed at a specific location and time in the SMC. With a normalization constant this spatially dependent star-formation rate is the prior on position and time:
\begin{equation}
P(\alpha_{\rm i}, \delta_{\rm i}, t_{\rm i}) = \frac{1}{N_{\rm SMC}} {\rm SFR}(\alpha_{\rm i}, \delta_{\rm i}, t_{\rm i}),
\end{equation}
where $N_{\rm SMC}$ is the number of stars with $Z=0.008$ produced throughout the lifetime of the SMC. Figure \ref{fig:SMC_SFH} shows the star-formation rate map for four different times spanning the range of typical HMXB lifetimes. These maps indicate that the locations of high star-formation regions have substantially evolved over the past 5$\times10^7$ years.





\subsection{Likelihood Functions: $P(x_{\rm obs} \given \vec{x}_{\rm i})$}


We can now define a likelihood function $P(x_{\rm obs} \given \vec{x}_{\rm i})$ which represents how likely the initial conditions are to produce a binary similar to the observations. 

When modeling a population of systems, the likelihood is  either unity or zero depending on whether $\vec{x}_{\rm f}$ represents a binary of interest:
\begin{equation}
P(x_{\rm type} | \vec{x}_{\rm i}, M) = 
\begin{cases}
1, & \vec{x}_{\rm f} \in x_{\rm type} \\
0, & \vec{x}_{\rm f} \notin x_{\rm type},
\end{cases}
\end{equation}
where $M$ is our binary evolution model. This can be intuitively understood. For example, if $\vec{x}_{\rm f}$ represents a double NS, it has a zero likelihood of being considered a HMXB, and vice versa. Note that, since $t_{\rm i}$ is a model parameter, 

Distrihas a zero likelihood of being considered a BHBH, even if it will ultimately evolve into a BHBH.b

When modeling a specific, individual system, the likelihood function may flexibly incorporate observational constraints. Instead of 


utions of tan HMXB he components of $\vec{x}_{\rm f}$ (such as the spatial distribution of binaries, X-ray luminosity function, orbital period distribution, etc.) can then provide model predictions for binary populations or comparisons to observational samples.




To generate our likelihood function, we now marginalize over four parameters, the true orbital parameters, $P_{\rm orb}$ and $e$, the true (current) companion mass $M_2$, and the systemic velocity ($v_{\rm sys}$). Our model likelihood then becomes:
\begin{eqnarray}
P(D \given \vec{x}_{\rm i}, M) &=& \int \dd P_{\rm orb}\ \dd e\ \dd M_2\ \dd v_{\rm sys} \nonumber \\
& & \qquad \times P( P_{\rm orb}, e, M_2, v_{\rm sys}, D \given \vec{x}_{\rm i}, M).
\end{eqnarray}
We substitute for $D$, and based on independence we factor the integrand into separate, tractable parts:
\begin{eqnarray}
P(D \given \vec{x}_{\rm i}, M) &=&  \int \dd P_{\rm orb}\ \dd e\ \dd M_2\ \dd v_{\rm sys}\ P(P'_{\rm orb} \given P_{\rm orb}) \nonumber \\
	& & \qquad \times P(e' \given e)\ P(M'_2 \given M_2) \nonumber \\
	& & \qquad \times P(P_{\rm orb}, e, M_2, v_{\rm sys} \given \vec{x}_{\rm i}) \nonumber \\
	& & \qquad \times P(\alpha, \delta \given \vec{x}_{\rm i}, v_{\rm sys}). \label{eq:marginalized}
\end{eqnarray}


The first three terms in the integrand of Equation \ref{eq:marginalized}, $P(P'_{\rm orb} \given P_{\rm orb})$, $P(e' \given e)$, and $P(M'_2 \given M_2)$, account for the observational uncertainties on the binary's orbit. We model these with Gaussian uncertainties. In the future, $P(M'_2 \given M_2)$ could be based on the photometric data for the donor star, which in combination with stellar evolution models, can constrain its mass. We discuss these three Gaussian uncertainties along with the fourth term in the integrand, which describes the function evolving the binary from its {\it ab initio} state to the $P_{\rm orb}$, $e$, $M_2$, and $v_{\rm sys}$ of the system today, in Section \ref{sec:likelihood_ind_binary}.

The last term in the integrand accounts for the fact that the system's birth place will, in general, be different from its observed position since the centre of mass of a system received a kick during the primary's core collapse. We explicitly include the dependence on $\vec{x_{\rm i}}$ and $v_{\rm sys}$ since the distance travelled depends on both the system's velocity and the time since the primary's SN. We derive this term in Section \ref{sec:ra_dec} below.



\subsection{Priors} \label{sec:priors_indiv}


\begin{figure}
\begin{center}
\includegraphics[width=0.95\columnwidth]{fig2.pdf}
\caption{The prior on both position of the binary's birth location and time depends on the star-formation history. Only the star-formation history within the cone, the size and shape of which is set by the binary parameters, are taken into account. We show contours indicating regions of high (red) and low (blue) star-formation for three different ages. Determining the prior on the star-formation history involves integrating the star-formation history throughout the cone.}
\label{fig:prior_SFH}
\end{center}
\end{figure}


The binary evolution prescriptions and SN kick priors are the same as described in Section \ref{sec:priors}. Only the star-formation rate {\it normalization} for the prior on $\alpha_{\rm i}$, $\delta_{\rm i}$, and $t_{\rm i}$ differs; Figure \ref{fig:prior_SFH} demonstrates the source of the difference.


An object at a current location $(\alpha, \delta)$ could have been formed only within a region (bounded by circles) that progressively increases for older birth times since older systems have had more time to travel from their birth locations. Only stars formed within this cone contribute to the normalization constant; stars formed outside the cone could not have led to the observed system. The shape and size of the cone changes depending on the binary parameters (the `slope' of the cone depends on $v_{\rm sys}$), therefore the prior needs to be recalculated for each set of model parameters. Because of the function calls, determining this normalization constant is the most computationally expensive calculation of this method. 

The normalization constant is determined by integrating over the spatially resolved star-formation rate throughout the cone and setting the quantity to unity:
\begin{equation}
1 = C_{\rm SFH}\ \int_{t_{\rm min}}^{t_{\rm max}} \int_0^{2 \pi} \int_0^{\theta_{\rm C}} \dd t_{\rm i}\ \dd \phi\ \dd \theta\ {\rm SFR}(\theta, \phi, t_{\rm i}). 
\end{equation}
where $\theta$ and $\phi$ are the polar and azimuthal angles (separate from similar parameters used to describe the SN kick direction) defining the cone. See Figure \ref{fig:prior_SFH} for the definition of these angles. $\theta_{\rm C}$ is the critical angle bounding the cone:
\begin{equation}
\theta_{\rm C} = \frac{v_{\rm sys} \left[ t_{\rm i} - t(M_{\rm 1,i}) \right]}{D_{\rm SMC}}, \label{eq:theta_c}
\end{equation}
where $t(M)$ is the lifetime of a star of mass $M$ and the distance to the SMC, $D_{\rm SMC}$, is 61 kpc \citep{hilditch05}. The bounds on the integral over $t_{\rm i}$ are set by the minimum and maximum times during which the system can emit X-rays: 
\begin{eqnarray}
t_{\rm min} &=& t(M_{\rm 1,i}) \\
t_{\rm max} &=& t(M_{\rm 2,MT}) - t_{\rm eff} + t_{\rm min},
\end{eqnarray}
where $t_{\rm eff}$ is defined in Equation \ref{eq:t_eff_1}.

We choose to calculate the integral using a Monte Carlo sampling technique. We draw $N$ random samples throughout the cone. The integral is the product of the average of the samples and the cone's volume:
\begin{equation}
\frac{1}{C_{\rm SFH}} \approx \frac{\pi \theta_{\rm C}^2 (t_{\rm max} - t_{\rm min})}{3N} \sum_j {\rm SFR}(\theta_j, \phi_j, t_{{\rm i},j}),
\end{equation}
where $(\theta_j, \phi_j, t_{{\rm i},j})$ are random samples, evenly distributed around the cone's volume. Formally, an infinite number of samples is required for the approximation above to become an equality. However, we find that for typical systems, 512 random samples provide a sufficiently accurate estimate of $C_{\rm SFH}$.



Random samples of $\phi$ are drawn from a uniform distribution between zero and $2\pi$. To obtain random samples of $\theta$ and $t$, we use inversion sampling, which involves obtaining random samples of the cumulative distribution function. The random values of $\theta$ and $t$ are the inversions of those samples:
 \begin{eqnarray}
\phi &\sim& U(0, 2\pi) \\
\theta &=& \sqrt{y_{\theta}} \theta_{\rm C}: y_{\theta} \sim U(0,1) \\
t &=& \sqrt[3]{y_{t}}\left( t_{\rm max} - t_{\rm min} \right) + t_{\rm min}: y_t \sim U(0,1).
\end{eqnarray}



\subsection{Binary Parameter Likelihood} \label{sec:likelihood_ind_binary}

Given a birth time and a particular set of initial binary parameters, the binary likelihood provides the probability that a binary with a particular $P'_{\rm orb}$, $e'$, and $M'_2$ will be formed. For HMXBs, determining this term relies first on the evolution from $\vec{x}_{\rm i}$ to $\vec{x}_{\rm f}$ described in Section \ref{sec:binary_evolve}. Our prescriptions for these three evolutionary states provide the term $P(P_{\rm orb}, e, M_2, v_{\rm sys} \given \vec{x}_{\rm i})$ in Equation \ref{eq:marginalized}. 

Then we need to compare with observations of the system. We will assume that the HMXBs in our population each have three separate observables: $M'_2$, $P'_{\rm obs}$, and $e'$, and that these each have Gaussian uncertainties with known standard deviations:
\begin{eqnarray}
P(e' \given e) &=& \mathcal{N}(e' ; e, \sigma^2_e) \\
P(P_{\rm orb}' \given P_{\rm orb}) &=& \mathcal{N}(P_{\rm orb}' ; P_{\rm orb}, \sigma^2_{P_{\rm orb}}) \\
P(M'_2 \given M_2) &=& \mathcal{N}(M'_2; M_2, \sigma^2_{M_2}).
\end{eqnarray}

Our method can adapt to any observation of individual stellar parameters included in our model; we only need to compare the evolved binary parameters to the observations.


\subsection{Position Likelihood} \label{sec:ra_dec}
The position likelihood provides the probability that, given a system's position, systemic velocity, and time since SN, the system would be observed at its current position. Since HMXBs are generally short lived, this probability is non-zero for only a small region around any given HMXB's observed position. To solve the positional component of Equation \ref{eq:marginalized}, we first marginalize over $\omega$, the angle between the line of sight vector to the birth location and the systemic velocity vector:
\begin{equation}
P(\alpha, \delta \given \vec{x}_{\rm i}, v_{\rm sys} ) = \int \dd \omega\ P(\alpha, \delta, \omega \given \vec{x}_{\rm i}, v_{\rm sys} ). \label{eq:P_pos_1}
\end{equation}

\begin{figure}
\begin{center}
\includegraphics[width=0.95\columnwidth]{fig3.pdf}
\caption{Our representation of the current position $(\alpha, \delta)$ in relation to its birth position $(\alpha_{\rm i}, \delta_{\rm i})$. The distance the system traveled is $d$, which has a projected separation $s$. We express this transportation as a function of $\theta_{\rm proj}$ and position angle, $\phi$. Note, for typical systems $d << D_{\rm SMC}$.}
\label{fig:position_projection}
\end{center}
\end{figure}


We next perform a coordinate transformation from the absolute positional coordinates $\alpha$ and $\delta$ to the relative angular separation, $\theta_{\rm proj}$, and the position angle, $\phi$, measured from the system's birth location. Figure \ref{fig:position_projection} shows our parameterization of the transformation from a system's birth location at $\alpha_{\rm i}$ and $\delta_{\rm i}$ to its current location at $\alpha$ and $\delta$. The determinant of the Jacobian matrix for this transformation is:
\begin{equation}
J_{\rm coor} = \left| \frac{\partial \theta_{\rm proj}}{\partial \alpha} \frac{\partial \phi}{\partial \delta} - \frac{\partial \theta_{\rm proj}}{\partial \delta}\frac{\partial \phi}{\partial \alpha}  \right|.
\end{equation}
These partial derivatives can be calculated by taking the derivatives of standard formulae for the angular separation and position angle of double stars. Equation \ref{eq:P_pos_1} now becomes:
\begin{eqnarray}
P(\alpha, \delta \given \vec{x}_{\rm i}, v_{\rm sys} ) &=& \int \dd \omega\ P(\theta_{\rm proj}, \phi, \omega \given \vec{x}_{\rm i}, v_{\rm sys} )\ J_{\rm coor} \nonumber \\
&=& \int \dd \omega\ P(\theta_{\rm proj} \given \omega,  \vec{x}_{\rm i}, v_{\rm sys} ) \nonumber \\
& & \qquad \times\ P(\phi)\ P(\omega)\ J_{\rm coor}, \label{eq:P_pos_2}
\end{eqnarray}
where we have separated terms based on independence. $\omega$ is a randomly chosen polar angle and $\phi$ is a randomly chosen azimuthal angle: 
\begin{eqnarray}
P(\omega) &=& \frac{\sin \omega} {2};\ \omega \in [0,\pi] \\
P(\phi) &=& \frac{1}{2 \pi};\ \phi \in [0, 2\pi].
\end{eqnarray}


The physical distance an HMXB travels is the product of $v_{\rm sys}$ and the time since the primary's core collapse:
\begin{equation}
d = v_{\rm sys} \left[t_{\rm i} - t(M_{\rm 1,i}) \right].
\end{equation}
Due to the relatively short lifetime of HMXBs, and therefore short travel distances relative to galactic scales, we ignore the effects of the SMC's gravitational potential and assume the HMXBs move in free space after receiving a kick. We can only observe the projection of $d$ onto the sky, $s = d \sin \omega$. We can also approximate $s$ as the product of $D_{\rm SMC}$ and $\theta_{\rm proj}$. After equating these two expressions for $s$ and solving for $\theta_{\rm proj}$, the first term of the integrand in Equation \ref{eq:P_pos_2} becomes a delta function:
\begin{equation}
P(\theta_{\rm proj} \given \omega, \vec{x}_{\rm i}, v_{\rm sys}) = \delta \left[G(\omega)\right], \label{eq:P_theta_proj}
\end{equation}
where:
\begin{equation}
G(\omega) = \theta_{\rm proj} - \frac{v_{\rm sys} \left[t_{\rm i} - t(M_{\rm 1,i}) \right] \sin \omega}{D_{\rm SMC}}.\end{equation}


With the delta function from Equation \ref{eq:P_theta_proj}, the integral in Equation \ref{eq:P_pos_2} can be reduced:
\begin{equation}
\int \dd \omega\ P(\phi) P(\omega) \delta \left[ G(\omega) \right]  J_{\rm coor}\  =\ \sum_j\ \frac{P(\omega_j^{\star}) P(\phi)  J_{\rm coor}}{ \left| \frac{ \dd G (\omega) }{\dd \omega} \right|_{\omega_j^*}},
\end{equation}
where the sum is over $\omega_j^*$, the $j$ roots of $G(\omega)$. There are two roots corresponding to whether the object is in front of or behind its birth location. This integral can now be evaluated analytically:
\begin{equation}
P(\alpha, \delta \given \vec{x}_{\rm i}, v_{\rm sys} ) =
\begin{cases} 
      0, & \theta_{\rm proj} \geq \theta_{\rm C}\\
     \frac{\tan \omega^*}{2 \pi \theta_{\rm C}}  J_{\rm coor}, & \theta_{\rm proj} < \theta_{\rm C} 
   \end{cases}
\end{equation}
where:
\begin{equation}
\omega^{\star} = \sin^{-1} \left[ \frac{\theta_{\rm proj}}{\theta_{\rm C}} \right].
\end{equation}




\end{comment}















\section{Testing with Mock Systems}
\label{sec:mock}

Particularly because of the multidimensional nature of binary population synthesis, along with the potential for any particular system to be formed from different evolutionary channels, extensive testing of our method is required. As a first test, we randomly choose three sets of initial conditions that produce HMXBs, evolve them forward using \bse, then attempt to recover the initial parameters using the current values of these parameters by adding uncertainties that mimic observational errors. Specifically, we test one system in which we observe only the companion mass, one in which we measure the orbital period, and the mass function, and one with a precisely measured companion mass, orbital period, and eccentricity. The exact values of the initial conditions and the observed quantities are provided in Table \ref{tab:mock}.



{\bf ADD A TABLE HERE WITH MOCK DATA}

{\bf ADD A FIGURE HERE SHOWING WE RECOVER MOCK DATA PARAMETERS}








\section{Results: Population of HMXBs} \label{sec:results_population}


\begin{figure*}
\begin{center}
\includegraphics[width=0.95\textwidth]{fig4.pdf}
\caption{ The covariances and 1D histograms of our posterior distribution that produces HMXBs in the SMC. Some of the posterior distributions, such as $e$ and $\phi_{\rm k}$, closely reflect the priors probabilities, however others, such as $v_{\rm k}$, $\theta_{\rm k}$, and $t_{\rm i}$, show substantial differences. The covariance between $\alpha_{\rm i}$ and $\delta_{\rm i}$ should closely follow the spatially resolved star-formation rate over the posterior distribution of birth times. That the distribution is patchy indicates the walkers seem to be able to explore the entire spatial parameter space. }
\label{fig:population_corner}
\end{center}
\end{figure*}



\begin{figure}
\begin{center}
\includegraphics[width=0.95\columnwidth]{fig5.pdf}
\caption{ Comparison between the star-formation rate in the SMC 12 and 40 Myr ago (blue background) and the distribution of the current positions (black contours) of the HMXB population as predicted by the model. The three contour levels define regions containing 25\%, 50\%, and 75\% of the data. The majority of the HMXB population is predicted to be around the centre of the SMC, however our model suggests a population of HMXBs should reside in a region to the southeast of the SMC bar. Comparison between the two panels indicates that these two HMXB populations correspond to temporally distinct episodes of star formation, thereby causing the two peaks in the posterior distribution of $t_{\rm i}$ in Figure \ref{fig:population_corner}. Contours define regions somewhat larger than the SMC bar, indicating that HMXBs may be found some distance from the most active regions of star-formation in the SMC. Note that at the distance to the SMC \citep[61 pc;][]{hilditch05}, 1 degree is $\approx$ 1 kpc.}
\label{fig:population_ra_dec}
\end{center}
\end{figure}



To explore the parameter space based on the posterior probability in Equation \ref{eq:bayes_pop} we use the MCMC code {\tt emcee} \citep{foreman-mackey13}. {\tt emcee} employs an affine-invariant ensemble sampler using multiple `walkers' in concert \citep{goodman10}. Our model uses 160 walkers when applied to an HMXB population. Since a large portion of the parameter space has a zero probability, it must be ensured that the walkers are initialized in non-zero-probability regions of parameter space. We discuss our procedure for initializing the walkers in detail in Appendix \ref{sec:initialize}. After initializing, we run {\tt emcee} for a `burn-in' of 10,000 steps. We then run our simulation for 50,000 steps to gain sufficient statistics. We check the chains to make sure they have converged; we find autocorrelation lengths of $\sim$100 steps across model parameters and a typical acceptance fraction of 8-10\%. 

Figure \ref{fig:population_corner} shows covariances between the posterior distribution of model parameters. That these covariances produce well defined regions for the majority of parameters suggests that our model is able to explore the relevant parameter space. In principle, the walkers could all be confined to a local minimum, and other minima in the parameter space could exist. However, in practice we find that {\tt emcee} is efficient at identifying multiple minima, if they exist, even when they are separated by modest likelihood barriers. 


Distributions over the initial masses, $M_{\rm 1,i}$ and $M_{\rm 2,i}$, strongly favor lower mass stars, expected from the strong weighting of the IMF toward lower masses. The covariance between the two mass posterior distributions is bounded on the top since the primary must be more massive than the secondary, and the bottom since in our model, mass-ratios smaller than 0.3 will lead to unstable mass transfer. The posterior distribution strongly favors systems with small initial orbital separations, $a_{\rm i}$, due to the strong weighting from the prior probability and since initially tighter binaries are more likely to stay bound after the SN. The eccentricity posterior distribution closely reflects the prior. 

The SN kick magnitude posterior peaks at a velocity $\approx$150 km s$^{-1}$, substantially lower than the input Maxwellian prior. The posterior distribution over the azimuthal kick angle, $\phi_{\rm k}$, is flat, mirroring the flat prior on this parameter; Equation \ref{eq:SN_e} shows that $\phi_{\rm k}$ only affects the post-SN orbital eccentricity. We also identify a large deviation from the prior distribution for the polar kick angle, $\theta_{\rm k}$; as expected most surviving binaries received a SN kick in the reverse of their orbital motion. The tail of the distribution in $\theta_{\rm k}$ extends toward small values, but systems surviving such prograde kicks are exceedingly rare. The covariance between $v_{\rm k}$ and $\theta_{\rm k}$ hints that if the prior distribution on $v_{\rm k}$ were pushed toward smaller velocities, more binaries may survive SNe with prograde kicks. 

The last three rows and columns in Figure \ref{fig:population_corner} show distributions over birth position and time. Distributions over $\alpha_{\rm i}$ and $\delta_{\rm i}$ are patchy which is expected since recent star-formation in the SMC is scattered. The two peaks in the bottom right histogram in Figure \ref{fig:population_corner} show that HMXBs formed through this channel in two broad episodes of star formation: at ages of $\sim$15 and $\sim$40 Myr. The covariance between $\alpha_{\rm i}$ and $t_{\rm i}$ indicates that these two episodes occurred in different regions of the SMC. Despite the non-contiguous nature of the star-formation history, {\tt emcee} is able to explore the entire region of recent star-formation indicating the ability for {\tt emcee} to explore more generally disparate regions of parameter space.  


\begin{figure}
\begin{center}
\includegraphics[width=0.99\columnwidth]{fig6.pdf}
\caption{ The top panel shows the distribution of orbital periods and eccentricities of the current population of HMXBs. The middle panel shows the distribution of companion masses and systemic velocities. Contours in the bottom panel show the angular distance travelled as a function of  overall system lifetime (solid black) and time since the SN (dashed red). }
\label{fig:smc_population_HMXB}
\end{center}
\end{figure}


By taking the posterior distribution of birth positions, determining the systemic velocity kicks these systems received and for how long they traveled, and applying a random direction to their systemic velocity, we can determine the current position distribution for HMXBs in the SMC. Figure \ref{fig:population_ra_dec} compares the distribution of current positions (black contours) with the star-formation history at 12 and 40 Myr (blue background). That the contours are somewhat more extended than the star forming distribution indicates that kick velocities can be substantial enough to move the HMXB population away from its birth position by a several arcminutes. Note that at the SMC's distance of 61 kpc \citep{hilditch05}, 1 degree $\approx$ 1 kpc.


By taking the posterior distribution of the model parameters and evolving the binaries forward, we can identify the distributions of binary parameters today. The top panel of Figure \ref{fig:smc_population_HMXB} shows the distribution of current binary orbital periods, $P_{\rm orb}$, and eccentricities, $e$; the bulk of systems have $P_{\rm orb}$ ranging from weeks to years and $e>0.5$. Although our code is not yet intended to reproduce detailed astrophysical characteristics of HMXBs, we note that the observed HMXBs in the SMC roughly match these characteristics. The middle panel shows the distribution of current companion (donor) masses in HMXBs, compared with systemic velocities. This shows that the most common binaries have $v_{\rm sys}\ \approx$ 20 km s$^{-1}$ and current donor star masses, $M_2$, ranging from 10-14 \Msun. This mass range is in agreement with the fact that the spectral type distribution of the donor stars in Be-type XRBs in the SMC peaks at B1$-$B2 \citep{mcbride08,maravelias14}. The bottom panel compares the angular separation between a binary's birth position and its current position with both its birth time (solid black contours) and the time since the SN (dashed red contours). As expected, systems with longer times since the SN have moved farther from their birth location. 




\section{Results: Individual HMXBs} \label{sec:results_individual}


To test the ability of our model to constrain the initial binary parameters for an arbitrary individual HMXB, we first create two separate mock systems with different sets of initial conditions (model parameter values). We choose values for these synthetic systems such that they will form X-ray bright binaries today given their birth times, and we choose birth locations with intense star-formation at the specified birth time. We then apply our forward evolution equations provided in Section \ref{sec:binary_evolve} to evolve the systems through stable mass transfer, SN, and the X-ray luminous phase. Finally, we `observe' the systems and record the output $\alpha$, $\delta$, $M'_2$, $P'_{\rm orb}$, and $e'$.\footnote{To get the current position from the assigned input birth position, we choose a random direction for the systemic velocity to move each system.} 


We apply a random Gaussian deviation of 1 \Msun, 1 day, and 0.05 to $M'_2$, $P'_{\rm orb}$, and $e'$, respectively, to simulate measurement uncertainties of these parameters. These five mock-observed quantities are the inputs for our statistical method outlined in Section \ref{sec:stats_individual}. This involves initializing the chains, which we describe in detail in Appendix \ref{sec:initialize}. After initializing, we run {\tt emcee} for four separate `burn-in' stages of 10,000 steps. We then run our simulation for 50,000 steps to gain sufficient statistics to determine the posterior probability distributions. Our model uses 80 walkers when applied to individual HMXBs.

Below, we describe how we first run our model on the two separate test binaries and then apply our model to the SMC binary PSR J0045$-$7319.


\begin{table}
\begin{center}
\caption{Input parameters for our two mock binaries described in Sections \ref{sec:test1} and \ref{sec:test2}.}
\label{tab:mock_data}
\begin{tabular}{lcc} 
\toprule
$\vec{x}_{\rm i}$ & Test 1 & Test 2 \\
\midrule
$M_{\rm 1,i}$ ($M_{\odot}$) & 13 & 9 \\
$M_{\rm 2,i}$ ($M_{\odot}$) & 10 & 4 \\
$a_{\rm i}$ ($R_{\odot}$) & 150 & 300 \\
$e_{\rm i}$ & 0.7 & 0.5 \\
$v_{\rm k}$ (km s$^{-1}$) & 100 & 100 \\
$\theta_{\rm k}$ (rad.) & 2.9 & 2.4 \\
$\phi_{\rm k}$ (rad.) & 0.9 & 1.2 \\
$\alpha_{\rm i}$ & 00:54:02.4 & 01:03:12.0 \\
$\delta_{\rm i}$ & -72:42:00.0 & -72:05:60.0 \\
$t_{\rm i}$ (Myr) & 22 & 40 \\
\bottomrule
\end{tabular}
\end{center}
\end{table}

\begin{table}
\begin{center}
\caption{The observed values for the two mock systems are shown in the first two columns, while the SMC system PSR J0045$-$7319 is in the last column. The observed values for PSR J0045$-$7319 are much more precise than listed here. We artificially increase the uncertainties on $P'_{\rm orb}$ and $e'$ to match the precision typical of HMXBs in the SMC.}
\label{tab:observations}
\begin{tabular}{lccc} 
\toprule
Parameter & Test 1 & Test 2 & J0045$-$7319 \\
\midrule
$\alpha$ & 00:54:00 & 01:03:36 & 00:45:35.26 \\
$\delta$ & $-$72:40:48 & $-$72:03:00 & $-$73:19:03.32 \\
$P'_{\rm orb}$ (days) & 50.55$\pm$1.0 & 616.52$\pm$1.0 & 51.169$\pm$1.0 \\
$e'$ & 0.467$\pm$0.05 & 0.305$\pm$0.05 & 0.808$\pm$0.05 \\
$M'_2$ (\Msun) & 19.2$\pm$1.0 & 10.7$\pm$1.0 & 8.8$\pm$1.8 \\
\bottomrule
\end{tabular}
\end{center}
\end{table}




\subsection{Test Case 1} \label{sec:test1}


The values of the 10 input parameters that define our first test case are provided in the second column of Table \ref{tab:mock_data}. This binary is relatively younger and more massive than the average in the HMXB distribution shown in Figure \ref{fig:population_corner}. We provide the present-day `observations' of this mock system, the parameters on which we run our model, in the second column of Table \ref{tab:observations}.

The bottom left panels in Figure \ref{fig:test1_corner} show our posterior probability distributions and their covariances when we run our model on this test system. Blue horizontal and vertical lines indicate the true values; each of these lines falls in a high probability portion of the parameter space, indicating our model can successfully recover the input model parameters. 

Some of the covariances between parameters can be easily understood. The current orbital shape is strongly affected by the SN kick. Therefore, structure in the posterior distributions for $v_{\rm k}$, $\theta_{\rm k}$, and $\phi_{\rm k}$ are to match the current observed $P'_{\rm orb}$ and $e'$. The posterior distributions for $M_{\rm 1,i}$ and $M_{\rm 2,i}$, and therefore $t_{\rm i}$, are determined primarily by the current companion mass. Since we assume the system circularizes due to the first mass transfer phase at the periapse separation, there is a strong covariance between $a_{\rm i}$ and $e_{\rm i}$.


\begin{figure*}
\begin{center}
\includegraphics[width=0.95\textwidth]{fig7.pdf}
\caption{Posterior probability distributions and their covariances when applied to test case 1. The true input values are indicated by the blue lines in each plot. Our model successfully reproduces all of these parameters. Structure in the distributions for $\theta_{\rm k}$ and $\phi_{\rm k}$ are to match the observed $P_{\rm orb}$ and $e$. The top right panel compares the star-formation rate (blue background) at 22 Myr, the input age of the system, with the posterior distribution of birth positions (black contours). The distribution roughly follows the region of high star-formation. The red star in the center of the contours denotes the system's current position. }
\label{fig:test1_corner}
\end{center}
\end{figure*}


In the top right panel of Figure \ref{fig:test1_corner}, we compare the birth position posterior distribution (contours0 to the star-formation history (blue background) around $\alpha_{\rm i}$ and $\delta_{\rm i}$ at 22 Myr, the system's true age. The posterior distribution of birth positions closely circles the system's current position (red star). This is unsurprising since the distribution in the lower panel of Figure \ref{fig:smc_population_HMXB} indicates that a system with an age of 22 Myr is likely to stay within 10\amin\ of its birth location.


\subsection{Test Case 2} \label{sec:test2}

\begin{figure*}
\begin{center}
\includegraphics[width=0.95\textwidth]{fig8.pdf}
\caption{Posterior probability distributions and their covariances when applied to test case 2. As in Figure \ref{fig:test1_corner}, the true input values are indicated by the blue lines in each plot. Our model again successfully reproduces all of these parameters. As in Figure \ref{fig:test1_corner} the top right panel compares the star-formation rate (blue background) with the posterior distribution of birth positions (black contours). Again, the posterior birth position distribution roughly follows the region of high star-formation. The model correctly identifies that the system's birth position is at somewhat smaller declinations than its current position, indicated by the red star in the top right panel.}
\label{fig:test2_corner}
\end{center}
\end{figure*}

The values of the 10 input parameters that define our second test case are provided in the third column of Table \ref{tab:mock_data}. Compared with our first test case, this binary is relatively older and less massive. The third column of Table \ref{tab:observations} shows the present day `observed' parameters of this test system. We run our model on these mock data, using the same burn-in procedure and number of steps as in our first test case.

We provide the posterior distribution of our model parameters (along with the true input values, again as blue lines) in Figure \ref{fig:test2_corner}. These distributions are not as smooth as those in the Figure \ref{fig:test1_corner} because our acceptance fraction for this model was somewhat lower. Nevertheless, our model is able to successfully recover most of the model input values. However the posterior distribution of $v_{\rm k}$ is highly skewed compared to the true value. 


Certain covariances between model parameters are seen in both test cases, for instance between $M_{\rm 1,i}$, $M_{\rm 2,i}$, and $t_{\rm i}$. The SN kick parameters have similar forms, but appear shifted and skewed between the two test cases; the exact form of the covariances between model parameters are strongly dependent upon each system's particular observed properties, as expected. 


\subsection{PSR J0045$-$7319}

The SMC pulsar binary, PSR J0045$-$7319, was first identified by \citet{ables87} with the Parkes 64-m radio telescope, but it was not until follow-up observations by \citet{kaspi94} that the pulsar was identified as a component of a 51-day orbit with an eccentricity of 0.808. These authors identify the optical companion as a B1 V star, indicating a mass of 10-12\Msun. Follow-up optical observations by \citet{bell95} identify and fit a radial velocity curve for the companion to the pulsar, finding a mass for the B star of 8.8$\pm$1.8\Msun. 

The NS in this binary is observed as a pulsar, allowing an extremely precise measurement of $P_{\rm orb}$ and $e$. However, most HMXBs lack such precise measurements, if they exist at all. Since we currently wish to demonstrate the ability of our method to derive constraints for typical HMXBs in the SMC and elsewhere, rather than actually provide astrophysical constraints for this particular system, we adopt uncertainties on these values of 1 day and 0.05 on $P_{\rm orb}$ and $e$, respectively. We provide the relevant observed parameters and their adopted uncertainties for the SMC binary PSR J0045$-$7319 in the fourth column of Table \ref{tab:observations}. For precise system parameters derived from radio observations, see \citet{kaspi94}.

The classification of PSR J0045$-$7319 as an HMXB is questionable since it has only been detected in X-rays once \citep{galache08}. Nevertheless, the system can be considered an HMXB progenitor \citep{kaspi94}, and it is possible that the system is emitting in X-rays with a luminosity below the sensitivity limit of most observations. Because its formation path is nearly identical to that of the rest of the HMXB population in the SMC, it provides us with an excellent test case for our method.

We apply our MCMC model to PSR J0045$-$7319. As was done previously in the two test cases, we first run four separate burn-ins stages for 10,000 steps each, and we check the chains to ensure they have converged. We then throw these chains away and run the MCMC sampler for 50,000 steps. Autocorrelation lengths are $\approx$100 and acceptance fractions are typically 3-5\%. 

The resulting posterior distributions of model parameters are shown in Figure \ref{fig:J0045_corner}. As with the test cases, the structure in the posterior distributions of $v_{\rm k}$, $\theta_{\rm k}$, and $\phi_{\rm k}$ are all the result of the model matching the observed $P_{\rm orb}$ and $e$. 




\begin{figure*}
\begin{center}
\includegraphics[width=0.95\textwidth]{fig9.pdf}
\caption{Posterior probability distributions and their covariances when applied to the SMC pulsar binary J0045$-$7319. The top right panel shows the posterior probability distributions for the birth position (black lines) compared with the star-formation rate (color background) 50 Myr ago at the peak posterior probability for the birth of J0045$-$7319 as indicated by our model. The current position of the system (red star) is in a region of low star-formation rate, and while the posterior distribution of birth positions from our model centers around the system's current position, the posterior distribution spreads to the northeast with high star-formation rates.}
\label{fig:J0045_corner}
\end{center}
\end{figure*}

The covariance between $\alpha_{\rm i}$ and $\delta_{\rm i}$ in Figure \ref{fig:J0045_corner} demonstrates the birth coordinate distribution for J0045$-$7319. In the top right panel of Figure \ref{fig:J0045_corner} we compare this distribution (black contours) to the star-formation rate (blue background) 50 Myr ago. The system (red star) currently lies in a region with little star-formation at its most likely birth age; the system must have travelled some distance from its birth location. However, the contours indicate that the system could not have travelled too far. The posterior probability spreads to the northeast which has somewhat higher star-formation rates.



\begin{figure*}
\begin{center}
\includegraphics[width=0.95\textwidth]{fig10.pdf}
\caption{Uncertainties in the observed values of $P_{\rm orb}$, $e$, and $M_2$ for PSR J0045$-$7319 are indicated by the Gaussian curves (dashed lines). Taking samples from the posterior distribution of model parameters and evolving them through our binary evolution prescriptions provides the posterior distribution of the observables (solid histogram). The posterior distribution occupies the same region as the observations in $P_{\rm orb}$ and $e$ space, evidence that the model has converged. Non-flat priors on the model parameters and binary evolution in general skew the distribution $M_2$ of the resulting HMXBs.}
\label{fig:J0045_forward}
\end{center}
\end{figure*}

In our two test cases, we compared the posterior distribution of model parameters to our input values. In addition to verifying the validity of our model, this comparison provides an additional check that the MCMC walkers have converged. Obviously, we cannot perform such a test for PSR J0045$-$7319. As an alternative, we evolve the posterior distribution of initial binary parameters through our binary evolution prescriptions and show the posterior probability of the observables $P_{\rm orb}$, $e$, and $M_2$ as solid histograms in Figure \ref{fig:J0045_forward}. Dashed lines indicate the observational uncertainties on the measurements (which form the binary likelihood functions defined in Section \ref{sec:likelihood_ind_binary}) of these three parameters. 


Assuming the model can accurately describe the observations, a converged model will have posterior distributions of model parameters similar to the observed values. Such a comparison provides an important check on both the feasibility of the model to adequately explain the data and the model's convergence. In general, it is not expected that the posterior distributions exactly match the uncertainties on the observables; non-flat prior distributions on the model parameters and binary evolution in general will skew the resulting posterior distribution of observables. However, the posterior and observed distributions over $P_{\rm orb}$ and $e$ are nearly identical. At the same time, the model predicts a somewhat larger mass for $M_2$ than the observations suggest. This is an example of how our model can improve the constraints on system characteristics, even those that are directly observable.






\section{Discussion and Conclusions}
\label{sec:discussion}

\subsection{A New Approach in BPS}

Past observations have shown that the population of extragalactic HMXBs are often found near, but not necessarily coincident with, regions of high star-formation \citep{zezas02b,kaaret04}. \citet{sepinsky05} showed that SN kicks can lead to a displacement between a binary's current position and its birth location. At the same time, the SN kick affects the binary's orbit, and a correlation should exist between binary parameters and the distance an HMXB travels from its birth site \citep{zuo10,zuo15}. These models use BPS to reproduce general characteristics of the HMXB population, such as the observation that HMXBs with higher X-ray luminosities tend to be found closer to star forming regions than systems with lower X-ray luminosities.


Traditional BPS is too inefficient to correlate the local star-formation history with binary characteristics for individual systems; too many of the Monte Carlo generated systems merge or disrupt at some point during their evolution. Of those that do evolve into HMXBs only a fraction evolve into systems of interest for a given study. The best-observed HMXBs, those systems with the most potential to constrain binary evolution, are often the least efficient to model since many separate observational characteristics need to be simultaneously matched.


In this work we describe a fully Bayesian method that allows for more detailed comparisons between HMXBs and star-formation histories. We interpret BPS as a model parameter problem that includes the spatially resolved star-formation history as a prior, together with observations of individual systems and their uncertainties in the likelihood function. Prior probabilities for the binary parameters and SN kicks are based on the same probabilities used by traditional BPS.


Our model involves several novel features: First, simultaneous consideration of source position and orbital parameters, which can set more stringent constraints on kick velocities. Second, an MCMC fitting approach to explore the parameter space, which provides more efficient sampling, particularly important for rare or short-lived systems. Third, flexible inclusion of different observational constraints or biases without the need for fine tuning of the methodology, allowing for a consistent approach to a heterogeneous data set. Finally, adaptability to study both individual systems as well as HMXB populations.


We run a simple benchmarking study to estimate the efficiency gain using this method. We run 5.5$\times 10^6$ binaries using our simplified binary evolution prescription with random initial parameters drawn from the same probabilities we use as priors for our model parameters. Of these, only 2.6$\times 10^5$, or $\approx$ 5\%, become HMXBs. We further select those binaries that have $M_2$, $P_{\rm orb}$, and $e$ within 3-$\sigma$ of the observed values of J0045$-$7319 and also with birth positions close enough to its current position to have moved towards it. Only 830 systems, or 0.015\%, satisfy these criteria. In our MCMC implementation, we run 7.2$\times 10^6$ different models (80 walkers with 40,000 burn-in steps followed by 50,000 steps to derive the posterior distributions) to produce approximately 4$\times 10^4$ independent samples after accounting for autocorrelation lengths of 100. To gain equivalent statistics using traditional binary evolution methods to model J0045$-$7319, at the rate above, $\approx$2.5$\times 10^8$ binaries would need to be simulated. Had we selected only those systems within 2-$\sigma$ of the observations, traditional BPS is even less efficient: only 249 of $5.5\times10^6$ systems, or 0.005\%, are consistent with forming J0045$-$7319. To emphasize this point, in their study of the ULX M82 X-2, \citet{fragos15} generated 10$^7$ binaries, identifying only 10$^2$-10$^3$ binaries consistent with the observations. The inefficiency of traditional methods limits the constraints we can set on this and other rare binary systems (such as double NSs, double black hole binaries, ULXs, etc.).



Our model avoids some of the common problems of traditional BPS. When comparing to a observed populations, traditional BPS studies often ignore observational uncertainties (or treat them in an ad hoc manner); whereas uncertainties are seamlessly included in our Bayesian formalism in the form of a likelihood function. Unique or rare systems may have been formed in relatively low probability regions of the parameter space, and using traditional BPS, it is both difficult to synthesize a statistically substantial number of systems and to know if those systems fully represent the parameter space. By moving through the parameter space based on the {\it posterior} probability, rather than making random draws from the prior probabilities, our MCMC model is able to efficiently generate a statistically significant distribution of posterior samples. This efficiency translates into the faster inclusion of updates to binary evolution physics and a faster comparison between XRB evolution models with different physical prescriptions. 





\subsection{Current Limitations and Future Directions}
\label{sec:limitations}


The potential applications for MCMC in BPS are numerous, as are the potential pitfalls. Careful attention needs to be paid to ensure the prior probabilities are properly described since binary parameter priors are often strong; the prior distributions should all be properly normalized, particularly when model parameters have joint priors such as with $M_{1,{\rm i}}$ and $M_{2,{\rm i}}$ or $\alpha_{\rm i}$, $\delta_{\rm i}$ and $t_{\rm i}$ in our model.


The posterior chains need to be converged to draw conclusions from the model. Although convergence cannot be guaranteed, we provide three informal methods to check for convergence in the chains: First, visual inspection should indicate that the chains asymptotically approach the same value with roughly identical variance. Second, the distributions of posterior parameters in the covariances between model parameters should be smoothly varying unless there are underlying physical reasons for abrupt changes in the distributions (such as non-contiguous star formation). Third, when evolving the posterior distribution of initial binary parameters through our binary evolution code, the posterior distribution of observables should roughly correspond to (but not necessarily exactly mimic) the values indicated by the observations and their uncertainties. In Figure \ref{fig:J0045_forward} we use this method to demonstrate that the chains are converged after applying our model to PSR J0045$-$7319. Using posterior samples from unconverged models can lead to incorrect results and conclusions.


A potential downside of MCMC methods is that they can have problems moving across sharp boundaries in parameter space. Since we follow only one evolutionary channel in this study, it remains to be seen if this will lead to problems with walkers transitioning between different evolutionary channels. We note here that our results in Section \ref{sec:results_population} indicate that the walkers successfully move across non-contiguous star forming regions, suggesting such sharp boundaries introduced by improvements in the binary evolution prescriptions may not be problematic. However, sharp boundaries in the binary evolution may reduce the acceptance fractions.


Another limitation of our model is that each HMXB must be calculated independently with our method, while with traditional BPS, the entire population need only be run once per model; the same synthetic population can be independently compared to each observed system.


The relative efficiency of our method compared to traditional BPS depends on the problem at hand. Traditional methods will be less efficient when the binary population is rare or short-lived and therefore the region of relevant parameter space is smaller. However, our method suffers from inefficiencies as well: each step is related to the previous one, therefore the set of independent posterior samples is reduced by a factor of the autocorrelation length. For the cases presented in this work, we find that autocorrelation lengths are typically $\approx$100. 


Although our method efficiently identifies the initial  parameters forming a specific HMXB (or population of HMXBs), this is only the first step. When combined with state-of-the-art binary evolution codes, the technique described in this work can be used to determine aspects of the system that are not observable directly, such as the binary’s formation scenario or whether it likely hosts a NS or black hole. Furthermore, the same MCMC techniques can be applied well-measured HMXBs in nearby galaxies using different binary evolution prescriptions. Ultimately, after combining similar analyses from multiple binaries, identifying the initial parameters forming the observed population can lead to constraints on underlying models for key binary evolution physics. For instance, by combining the posterior distributions of individual systems, we can set better constraints on the distribution of SN kick velocities.



Our model can be expanded in a number of ways. First, since we ultimately want to employ this model to derive astrophysical constraints on individual systems, more realistic binary evolution physics will need to be included, along with other evolutionary scenarios forming HMXBs. Second, we plan to expand this model to other nearby galaxies including the LMC which also has a well understood star-formation history \citep{harris09} and HMXB population \citep[e.g.,][]{antoniou16}. 


With the recent LIGO detections of binary black hole mergers \citep{abbott16a,abbott16b}, there is renewed interested in deriving evolutionary histories for merging evolved massive binaries \citep{belczynski16}. As instrument sensitivity improves and the number of detections increases, new methods for deriving evolutionary histories for merging compact objects will be crucial for fully understanding the formation of these systems and modeling their population. Although our model is derived for HMXBs, a subset of these systems, particularly those with massive Wolf-Rayet components that are unlikely to disrupt after the second component's core collapse, may evolve into binary black holes \citep{belczynski13,maccarone14}. With updates to the binary evolution physics, our MCMC approach could provide a natural, general method by which to efficiently derive posterior distributions for those systems forming merging black hole binaries. 



\subsection{Conclusions}
\label{sec:conclusions}


We describe a fully Bayesian method to identify the binary initial conditions forming both individual HMXBs and HMXB populations. Our method includes both binary evolution physics as well as spatially resolved star-formation histories to constrain the formation channels of HMXBs. As a test, we apply our method to two individual mock binaries with parameters and uncertainties typical of the observed HMXBs in the SMC. Our method is generally able to recover all the initial parameters of these binaries. Finally, we apply our model to the SMC pulsar binary PSR J0045$-$7319, and our model converges on the region of parameter space forming this binary; the posterior distribution of model parameters produces HMXBs matching the observations of PSR J0045$-$7319. Our model, and MCMC techniques more generally, has the potential to become a powerful tool for the study of binary populations.







\section*{Acknowledgements}
It is a pleasure to thank Vicky Kalogera and Stephen Justham for useful conversations. J.J.A. and A.Z. acknowledge funding from the European Research Council under the European Union's Seventh Framework Programme (FP/2007-2013)/ERC Grant Agreement n. 617001. T.F. acknowledges support from the Ambizione Fellowship of the Swiss National Science Foundation (grant PZ00P2-148123). We acknowledge use of the Metropolis HPC Facility at the CCQCN Center of the University of Crete, supported by the European Union Seventh Framework Programme (FP7-REGPOT-2012-2013-1) under grant agreement no. 316165.


\bibliographystyle{mnras}
\bibliography{references}


\appendix

\section{Walker initialization and Burn-in} \label{sec:initialize}

Only a small portion of the ten-dimensional parameter space has a non-zero posterior probability, and it is necessary to initialize the walkers in this small region. The most important parameters are $M_{\rm 1,i}$, $M_{\rm 2,i}$, and $t_{\rm i}$, since these are necessary to match the observed companion mass, $M'_2$. To initialize these parameters, we create and run a separate three-parameter MCMC model prior to running our larger ten-parameter model. 


% \begin{figure*}
% \begin{center}
% \includegraphics[width=0.95\textwidth]{figA1.pdf}
% \caption{The five columns show the evolution of chains from the four separate burn-in stages and the production stage from our test case 2, described in Section \ref{sec:test2}. The ten rows correspond to the ten parameters in $\vec{x}_{\rm i}$ in the order listed in Equation \ref{eq:x_i}. These chains show the typical burn-in time for our models. }
% \label{fig:burn-in_test2}
% \end{center}
% \end{figure*}


We solve the following Bayesian posterior probability:
\begin{equation}
P(M_{\rm 1,i}, M_{\rm 2,i}, t_{\rm i} \given M'_2) = \frac{ P(M'_2 \given M_{\rm 1,i}, M_{\rm 2,i}, t_{\rm i} ) P( M_{\rm 1,i}, M_{\rm 2,i}, t_{\rm i} )}{P(M'_2)}. \label{eq:appendix_1}
\end{equation}
The priors on $M_{\rm 1,i}$, $M_{\rm 2,i}$, and $t_{\rm i}$ are the same as provided above in Section \ref{sec:priors}. The likelihood distribution, $P(M'_2 \given M_{\rm 1,i}, M_{\rm 2,i}, t_{\rm i} )$, is adapted from our method described in Section \ref{sec:binary_evolve}. Since we are only interested in placing walkers in a non-zero part of parameter space, we ignore uncertainties on $M'_2$. The posterior distribution for our three-component model is the numerator of the right hand side of Equation \ref{eq:appendix_1}. 


This three-component model itself requires initialization of the model parameters. For each walker, we select random values of the three parameters from a limited range until the posterior probability is non-zero. We then run the model for 100 steps and take the final position of each walker as initial positions in $M_{\rm 1,i}$, $M_{\rm 2,i}$, and $t_{\rm i}$ parameter space. In practice, we find 100 steps sufficiently spread the walkers throughout the allowable region of this three-dimensional parameter space while requiring negligible computational time.


We set the initial values of $M_{\rm 1,i}$, $M_{\rm 2,i}$, and $t_{\rm i}$ for the walkers in our full ten-component model to the last values of the walkers in our three-component model. We randomly generate positions for the other seven parameters until the posterior probability is non-zero. If, after a certain number of iterations for a particular walker, we do not find any non-zero set of parameters, we move that walker by hand close to a walker with a non-zero posterior probability. 


Once the walkers have been set, we begin the burn-in stage. For the HMXB population described in Section \ref{sec:results_population}, this involves running the model for 10,000 steps so the walkers distribution around the parameter space based on the posterior probability. We find this is a sufficient number for the chains to converge. 


For our models for individual HMXBs described in Section \ref{sec:results_individual}, we run four separate burn-in stages. Between each of these, we move the ten lowest probability walkers to random positions within $\sim$0.5\% the highest likelihood walker. Then we restart the simulation, using the position of these walkers as the new initialized position. Figure \ref{fig:burn-in_test2} shows how the distribution of walkers evolves for the ten model parameters through the burn-in and production stages for our test case 2 described in Section \ref{sec:test2}. The input values for the parameters are indicated by the horizontal red line. This figure demonstrates why multiple burn-in stages are necessary: some of the walkers (especially $v_{\rm k}$) can become stuck in regions of low probability. The problem of walkers becoming stuck in low probability regions is particularly acute for our test case 2; while models for test case 1 and PSR J0045$-$7319 largely avoid this effect. Nevertheless, for consistency, we run multiple burn-in stages for these, too.


Our method of moving the walkers between burn-in stages shifts them to viable regions of parameter space. The last burn-in stage distributes the walkers throughout the parameter space, and we do not move any of the walkers between the last burn-in and our production stage. Our production run of 50,000 steps, shown in the rightmost column of Figure \ref{fig:burn-in_test2}, demonstrates that after these four burn-in stages, the walkers have converged. Conclusions are only drawn from the production stage, data from the burn-in stages are removed.




\section{{\tt BSE} in {\tt Python}} \label{sec:pyBSE}





\section{Supernova Kicks} \label{sec:trans_SN}

Prior to its core collapse, the primary star has survived for some time as a naked He-star with substantially enhanced winds \cite{belczynski10}. We ignore any accretion of these winds by the secondary, so mass loss tends to expand the binary's orbit. Our {\tt sse} interpolation provides the pre-SN He-star mass, $M_{\rm pre-SN}$ as a function of $M_{\rm MT}$. Using Jeans mode mass loss we can determine the pre-SN orbital separation, $a_{\rm pre-SN}$:
\begin{equation}
a_{\rm pre-SN} = \frac{M_{\rm 1, MT} + M_{\rm 2, MT}}{M_{\rm 1, pre-SN} + M_{\rm 2, MT}} a_{\rm MT}
\end{equation}


We next calculate the post-SN orbital separation, $a_{\rm SN}$, systemic velocity, $v_{\rm sys}$, and eccentricity, $e_{\rm SN}$, based on the equations in \citet{hills83} and \citet{kalogera96}. In our model, all core collapsing stars form 1.35 \Msun\ NSs. \citet{timmes96} suggest the maximum initial mass forming a NS is $\sim$20 \Msun, instead of the maximum mass of 30 \Msun\ we choose in this work; however as we will see below, our results are largely independent of this choice. Energy conservation provides $a_{\rm SN}$:
\begin{equation}
a_{\rm SN} = \left[ \frac{2 }{a_{\rm pre-SN}}  - \frac{v_1^2}{G(M_{\rm NS} + M_{\rm 2, MT})} \right]^{-1}. \label{eq:SN_A} \\
\end{equation}
where $v_1$ is the post-kick velocity of the primary (in the reference frame of an initially stationary secondary):
\begin{equation}
v_1^2 = 2v_{\rm k} v_{\rm r} \cos \theta_{\rm k} + v_{\rm k}^2 + v_{\rm r}^2, \label{eq:v_1}
\end{equation}
where $\theta_{\rm k}$ defines the angle between the kick velocity and the direction of orbital motion. Here, we have made the additional substitution:
\begin{equation}
v_{\rm r} = \sqrt{\frac{G (M_{\rm 1, pre-SN} + M_{\rm 2, MT})}{a_{\rm pre-SN}}}. \label{eq:v_r}
\end{equation}
The systemic velocity is:
\begin{equation}
v_{\rm sys}^2 = \beta^2 v_{\rm k}^2
   + v_{\rm r}^2 \left( \alpha - \beta \right)^2
   + 2 \beta v_{\rm r} v_{\rm k} \cos \theta_{\rm k} \left( \alpha - \beta \right)
    \label{eq:SN_v_sys}
\end{equation}
where we have included two substitutions:
\begin{eqnarray}
\alpha &=& \frac{M_{\rm 1, pre-SN}}{M_{\rm 1, pre-SN} + M_{\rm 2, MT}} \\
\beta &=& \frac{M_{\rm NS}}{M_{\rm NS} + M_{\rm 2, MT}}
\end{eqnarray}


The post-SN eccentricity is determined by angular momentum conservation:
\begin{eqnarray}
1-e_{\rm SN}^2 &=& \frac{a_{\rm pre-SN}^2}{a_{\rm SN}\ G (M_{\rm NS} + M_{2, MT})} \nonumber \\
 & & \times \left( v_{\rm k}^2 \cos^2\theta_{\rm k} + v_{\rm k}^2 \sin^2 \theta_{\rm k} \sin^2 \phi_{\rm k} \right. \nonumber \\
 & & \qquad + \left. 2 v_{\rm k} v_{\rm r} \cos \theta_{\rm k} + v_{\rm r}^2  \right). \label{eq:SN_e}
\end{eqnarray}
Systems with $0 \leq e < 1$ remain bound. It should be noted that $\phi_{\rm k}$ only affects the post-SN orbit through the $\sin^2 \phi_{\rm k}$ term when solving for the post-SN orbital eccentricity in Equation \ref{eq:SN_e}. 







\label{lastpage}

\end{document}
